\documentclass[xcolor=pdflatex,dvipsnames,table]{beamer}
\usepackage{epsfig,graphicx}
\usepackage{palatino}
\usepackage{fancybox}
\usepackage{relsize}
\usepackage[procnames]{listings}
\usepackage{hyperref}
\usepackage{qtree} % needed?
\usepackage{booktabs}
\usepackage{dirtree}
\usepackage[normalem]{ulem}


% fatter TT font
\renewcommand*\ttdefault{txtt}
% another TT, suggested by Alex
% \usepackage{inconsolata}
% \usepackage[T1]{fontenc} % needed as well?


\newcommand{\scale}{0.7}

\newcommand{\todo}[1]{{\emph{TODO: #1}}}
\newcommand{\martin}[1]{{\color{blue} Martin: #1}}
\newcommand{\abcdef}[1]{{\color{red} Author2: #1}}

% uncomment following for final submission
%\renewcommand{\todo}[1]{}
%\renewcommand{\martin}[1]{}
%\renewcommand{\author2}[1]{}

\newcommand{\code}[1]{{\texttt{#1}}}

\hypersetup{
  linkcolor  = black,
%  citecolor  = blue,
  urlcolor   = blue,
  colorlinks = true,
}

\beamertemplatenavigationsymbolsempty
\setbeamertemplate{footline}[frame number]





\newif\ifbook
% shared in slides and book

\lstdefinelanguage{chisel}{
  morekeywords={abstract,case,catch,class,def,%
    do,else,extends,false,final,finally,%
    for,if,implicit,import,match,mixin,%
    new,null,object,override,package,%
    private,protected,requires,return,sealed,%
    super,this,throw,trait,true,try,%
    type,val,var,while,with,yield},
  otherkeywords={=>,<-,<\%,<:,>:,\#,@},
  sensitive=true,
  morecomment=[l]{//},
  morecomment=[n]{/*}{*/},
  morestring=[b]",
  morestring=[b]',
  morestring=[b]"""
}

\usepackage{color}
\definecolor{dkgreen}{rgb}{0,0.6,0}
\definecolor{gray}{rgb}{0.5,0.5,0.5}
\definecolor{mauve}{rgb}{0.58,0,0.82}

% Default settings for code listings
\ifbook
\lstset{%frame=lines,
  language=chisel,
  aboveskip=3mm,
  belowskip=3mm,
  showstringspaces=false,
  columns=fixed, % basewidth=\mybasewidth,
  basicstyle={\small\ttfamily},
  numbers=none,
  numberstyle=\footnotesize,
  % identifierstyle=\color{red},
  breaklines=true,
  breakatwhitespace=true,
  procnamekeys={def, val, var, class, trait, object, extends},
  % procnamestyle=\ttfamily,
  tabsize=2,
  float
}
\else
\lstset{%frame=lines,
  language=chisel,
  aboveskip=3mm,
  belowskip=3mm,
  showstringspaces=false,
  columns=fixed, % basewidth=\mybasewidth,
  basicstyle={\small\ttfamily},
  numbers=none,
  numberstyle=\footnotesize\color{gray},
  % identifierstyle=\color{red},
  keywordstyle=\color{blue},
  commentstyle=\color{dkgreen},
  stringstyle=\color{mauve},
  breaklines=true,
  breakatwhitespace=true,
  procnamekeys={def, val, var, class, trait, object, extends},
  procnamestyle=\ttfamily\color{red},
  tabsize=2,
  float
}
\fi

\lstnewenvironment{chisel}[1][]
{\lstset{language=chisel,#1}}
{}

\newcommand{\shortlist}[1]{{\lstinputlisting[nolol]{#1}}}

\newcommand{\longlist}[3]{{\lstinputlisting[float, caption={#2}, label={#3}, frame=tb, captionpos=b]{#1}}}

\newcommand{\verylonglist}[3]{{\lstinputlisting[caption={#2}, label={#3}, frame=tb, captionpos=b]{#1}}}


% TikZ for diagrams
\usepackage{tikz}
\usetikzlibrary{positioning, arrows.meta}

% Optional visual tweaks
\tikzset{
    >=Stealth,
    every node/.style={rounded corners=2pt}
}

\title{Mixed Topics}
\author{Martin Schoeberl}
\date{\today}
\institute{Technical University of Denmark\\
Embedded Systems Engineering}

\begin{document}

\begin{frame}
\titlepage
\end{frame}




\begin{frame}[fragile]{Outline}
\begin{itemize}

\item xxx

\end{itemize}
\end{frame}


\begin{frame}[fragile]{Use a \code{var} for a Generator}
\begin{itemize}
\item A \code{var} can be reassigned
\item Just keep the Chisel components connected
\item No need to keep the references around
\begin{chisel}
class Delay(n: Int) extends Module {
    val io = IO(new Bundle {
        val in  = Input(Bool())
        val out = Output(Bool())
    })
    require(n >= 0)
    var lastConn = io.in
    for (i <- 0 until n)
        lastConn = RegNext(lastConn)
    io.out := lastConn
}
\end{chisel}
\end{itemize}
\end{frame}

\begin{frame}[fragile]{Do you want Recursion?}
\begin{itemize}
\item With a helper function
\begin{chisel}
class DelayRec(n: Int) extends Module {
    val io = IO(new Bundle {
        val in  = Input(Bool())
        val out = Output(Bool())
    })
    require(n >= 0)
    def helper(n: Int, lastConn: Bool): Bool = {
        if (n == 0) lastConn
        else helper(n-1, RegNext(lastConn))
    }
    io.out := helper(n, io.in)
}
\end{chisel}
\end{itemize}
\end{frame}

\begin{frame}[fragile]{Parameterize Number of Ports}
\begin{itemize}
\item Use a \code{Vec} 
\begin{chisel}
class Example(n: Int, w: Int) extends Module {
    val io = IO(new Bundle {
        val in  = Input(Vec(n, UInt(w.W)))
        val out = Output(UInt(w.W))
    })
\end{chisel}
\end{itemize}
\end{frame}



\begin{frame}[fragile]{Use Methods}
\begin{itemize}
\item Modules and Bundles are Scala classes
\item One can define methods on them
\item We have seen \code{apply()} on the companion object
\item Remember what this gave us?
\end{itemize}
\end{frame}

\begin{frame}[fragile]{Factory Method \code{apply}}
\begin{itemize}
\item Simpler component creation and use
\item Usage similar to built-in components, such as \code{Mux}
\end{itemize}
\begin{chisel}
val myAdder = Adder(x, y)
\end{chisel}
\begin{itemize}
\item A little bit more work on the component side
\item Define an \code{apply} method on the companion object that returns the component (output port)
\end{itemize}
\begin{chisel}
object Adder {
  def apply(a: UInt, b: UInt) = {
    val adder = Module(new Adder)
    adder.io.a := a
    adder.io.b := b
    adder.io.result
  }
}
\end{chisel}
\end{frame}

\begin{frame}[fragile]{Decoupled Components}
\begin{itemize}
\item Using the read/valid handshake
\item Some flexibility on sending and receiving
\item Data is transferred when ready and valid are asserted
\item Chisel provides a standard bundle for this: \code{DecoupledIO}
\shortlist{../code/fifo_decoupled.txt}
\end{itemize}
\end{frame}

\begin{frame}[fragile]{Methods on \code{DecoupledIO}}
\begin{itemize}
\item \code{fire} is true if and only if ready and valid are asserted
\item \code{enq(data)} sets data and sets valid to true (no check on ready)
\item \code{noenq} deasserts valis.
\item \code{deq/nodeq} for the receiver side
\begin{chisel}
when(io.in.fire()) {
  // do the transfer
}
\end{chisel}
\end{itemize}
\end{frame}

\begin{frame}[fragile]{FIFO Queue}
\begin{figure}
  \centering
  \includegraphics[scale=\scale]{../figures/fifo}
  \caption{A writer, a FIFO buffer, and a reader.}
  \label{fig:fifo}
\end{figure}
\begin{itemize}
\item FIFO style queues between a sender and a receiver
\item The sender/write enqueues/writes into the Queue
\item The receiver/reader dequeues/reads from the Queue
\item Can \emph{even out} bursty traffic
\end{itemize}
\end{frame}

\begin{frame}[fragile]{Queues}
\begin{itemize}
\item Used when there is bursty traffic
\begin{itemize}
\item E.g., on a serial port
\item Write faster from CPU send can send
\item Receive data when CPU is busy with other stuff
\end{itemize}
\item But can fill up when the sender is \emph{faster}
\item Showing implementation variations in the Chisel book
\begin{itemize}
\item Bubble FIFO (was proposed in lab 5)
\item double buffer, memory-based, ...
\end{itemize}
\end{itemize}
\end{frame}

\begin{frame}[fragile]{Bubble FIFO Example}
\begin{itemize}
\item FIFO interface
\shortlist{../code/fifo_io.txt}
\item Abstract base class
\shortlist{../code/fifo_abstract.txt}
\item \href{https://github.com/schoeberl/chisel-book/blob/master/src/main/scala/fifo/fifo.scala}{FIFO code}
\end{itemize}
\end{frame}

\begin{frame}[fragile]{Bubble FIFO}
\begin{itemize}
\item Simple, easy to understand
\item Uses minimal resources
\item However, each buffer stage has to toggle between empty and full
\item Maximum bandwidth is one word every two clock cycles
\item When full, needs N cycles for a restart
\begin{itemize}
\item The free element \emph{bubbles} towards the input
\end{itemize}
\item Better solutions
\begin{itemize}
\item Double buffer
\item Memory with read and write pointer
\end{itemize}
\item See Section 11.3 in the Chisel book
\end{itemize}
\end{frame}

\begin{frame}[fragile]{Chisel \code{Queue}}
\begin{itemize}
\item Chisel has a \code{Queue}
\item Interfaces are \code{DecoupledIO}s
\item Specify type and number of entries Queue(UInt(4.W), 8)
\item Optional arguments
\begin{itemize}
\item \code{pipe}: if full, allow concurrent enqueue and dequeue
\item \code{flow}: if empty, enqueued value available immediately for dequeue
\end{itemize}
\end{itemize}
\begin{chisel}
val q = Module(new Queue(UInt(), 16))
q.io.enq <> producer.io.out
consumer.io.in <> q.io.deq
\end{chisel}
\end{frame}

\begin{frame}[fragile]{Next Week}
\begin{itemize}
\item Guest lecture by Hans Jakob on
\item Chisel in Research
\end{itemize}
\end{frame}

\begin{frame}[fragile]{Summary}
\begin{itemize}
\item xxx
\end{itemize}
\end{frame}


\end{document}

\begin{frame}[fragile]{XXX}
\begin{itemize}
\item xxx
\item xxx
\item xxx
\begin{itemize}
\item xxx
\end{itemize}
\item xxx
\item xxx
\end{itemize}
\end{frame}

\begin{frame}[fragile]{Title}
\begin{itemize}
\item abc
\end{itemize}
\end{frame}
