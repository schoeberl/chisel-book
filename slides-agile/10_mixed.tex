\documentclass[xcolor=pdflatex,dvipsnames,table]{beamer}
\usepackage{epsfig,graphicx}
\usepackage{palatino}
\usepackage{fancybox}
\usepackage{relsize}
\usepackage[procnames]{listings}
\usepackage{hyperref}
\usepackage{qtree} % needed?
\usepackage{booktabs}
\usepackage{dirtree}
\usepackage[normalem]{ulem}


% fatter TT font
\renewcommand*\ttdefault{txtt}
% another TT, suggested by Alex
% \usepackage{inconsolata}
% \usepackage[T1]{fontenc} % needed as well?


\newcommand{\scale}{0.7}

\newcommand{\todo}[1]{{\emph{TODO: #1}}}
\newcommand{\martin}[1]{{\color{blue} Martin: #1}}
\newcommand{\abcdef}[1]{{\color{red} Author2: #1}}

% uncomment following for final submission
%\renewcommand{\todo}[1]{}
%\renewcommand{\martin}[1]{}
%\renewcommand{\author2}[1]{}

\newcommand{\code}[1]{{\texttt{#1}}}

\hypersetup{
  linkcolor  = black,
%  citecolor  = blue,
  urlcolor   = blue,
  colorlinks = true,
}

\beamertemplatenavigationsymbolsempty
\setbeamertemplate{footline}[frame number]





\newif\ifbook
% shared in slides and book

\lstdefinelanguage{chisel}{
  morekeywords={abstract,case,catch,class,def,%
    do,else,extends,false,final,finally,%
    for,if,implicit,import,match,mixin,%
    new,null,object,override,package,%
    private,protected,requires,return,sealed,%
    super,this,throw,trait,true,try,%
    type,val,var,while,with,yield},
  otherkeywords={=>,<-,<\%,<:,>:,\#,@},
  sensitive=true,
  morecomment=[l]{//},
  morecomment=[n]{/*}{*/},
  morestring=[b]",
  morestring=[b]',
  morestring=[b]"""
}

\usepackage{color}
\definecolor{dkgreen}{rgb}{0,0.6,0}
\definecolor{gray}{rgb}{0.5,0.5,0.5}
\definecolor{mauve}{rgb}{0.58,0,0.82}

% Default settings for code listings
\ifbook
\lstset{%frame=lines,
  language=chisel,
  aboveskip=3mm,
  belowskip=3mm,
  showstringspaces=false,
  columns=fixed, % basewidth=\mybasewidth,
  basicstyle={\small\ttfamily},
  numbers=none,
  numberstyle=\footnotesize,
  % identifierstyle=\color{red},
  breaklines=true,
  breakatwhitespace=true,
  procnamekeys={def, val, var, class, trait, object, extends},
  % procnamestyle=\ttfamily,
  tabsize=2,
  float
}
\else
\lstset{%frame=lines,
  language=chisel,
  aboveskip=3mm,
  belowskip=3mm,
  showstringspaces=false,
  columns=fixed, % basewidth=\mybasewidth,
  basicstyle={\small\ttfamily},
  numbers=none,
  numberstyle=\footnotesize\color{gray},
  % identifierstyle=\color{red},
  keywordstyle=\color{blue},
  commentstyle=\color{dkgreen},
  stringstyle=\color{mauve},
  breaklines=true,
  breakatwhitespace=true,
  procnamekeys={def, val, var, class, trait, object, extends},
  procnamestyle=\ttfamily\color{red},
  tabsize=2,
  float
}
\fi

\lstnewenvironment{chisel}[1][]
{\lstset{language=chisel,#1}}
{}

\newcommand{\shortlist}[1]{{\lstinputlisting[nolol]{#1}}}

\newcommand{\longlist}[3]{{\lstinputlisting[float, caption={#2}, label={#3}, frame=tb, captionpos=b]{#1}}}

\newcommand{\verylonglist}[3]{{\lstinputlisting[caption={#2}, label={#3}, frame=tb, captionpos=b]{#1}}}


% TikZ for diagrams
\usepackage{tikz}
\usetikzlibrary{positioning, arrows.meta}

% Optional visual tweaks
\tikzset{
    >=Stealth,
    every node/.style={rounded corners=2pt}
}

\title{Mixed Topics}
\author{Martin Schoeberl}
\date{\today}
\institute{Technical University of Denmark\\
Embedded Systems Engineering}

\begin{document}

\begin{frame}
\titlepage
\end{frame}




\begin{frame}[fragile]{Outline}
\begin{itemize}

\item Mixed collection of generator ideas
\item My last lecture (probably)
\item Today, intermediate project presentation and demo
\item Next week, another guest
\item Week 12: project work
\item Week 13: Final project presentation and demonstration

\end{itemize}
\end{frame}


\begin{frame}[fragile]{Use a \code{var} for a Generator}
\begin{itemize}
\item A \code{var} can be reassigned
\item Just keep the Chisel components connected
\item No need to keep the references around
\begin{chisel}
class Delay(n: Int) extends Module {
    val io = IO(new Bundle {
        val in  = Input(Bool())
        val out = Output(Bool())
    })
    require(n >= 0)
    var lastConn = io.in
    for (i <- 0 until n)
        lastConn = RegNext(lastConn)
    io.out := lastConn
}
\end{chisel}
\end{itemize}
\end{frame}

\begin{frame}[fragile]{Do you want Recursion?}
\begin{itemize}
\item With a helper function
\begin{chisel}
class DelayRec(n: Int) extends Module {
    val io = IO(new Bundle {
        val in  = Input(Bool())
        val out = Output(Bool())
    })
    require(n >= 0)
    def helper(n: Int, lastConn: Bool): Bool = {
        if (n == 0) lastConn
        else helper(n-1, RegNext(lastConn))
    }
    io.out := helper(n, io.in)
}
\end{chisel}
\end{itemize}
\end{frame}

\begin{frame}[fragile]{Parameterize Number of Ports}
\begin{itemize}
\item Use a \code{Vec} 
\begin{chisel}
class Example(n: Int, w: Int) extends Module {
    val io = IO(new Bundle {
        val in  = Input(Vec(n, UInt(w.W)))
        val out = Output(UInt(w.W))
    })
\end{chisel}
\end{itemize}
\end{frame}



\begin{frame}[fragile]{Use Methods}
\begin{itemize}
\item Modules and Bundles are Scala classes
\item One can define methods on them, like in any Scala class
\item We have seen \code{apply()} on the companion object
\item Remember what this gave us?
\end{itemize}
\end{frame}

\begin{frame}[fragile]{Factory Method \code{apply}}
\begin{itemize}
\item Simpler component creation and use
\item Usage similar to built-in components, such as \code{Mux}
\end{itemize}
\begin{chisel}
val myAdder = Adder(x, y)
\end{chisel}
\begin{itemize}
\item A little bit more work on the component side
\item Define an \code{apply} method on the companion object that returns the component (output port)
\end{itemize}
\begin{chisel}
object Adder {
  def apply(a: UInt, b: UInt) = {
    val adder = Module(new Adder)
    adder.io.a := a
    adder.io.b := b
    adder.io.result
  }
}
\end{chisel}
\end{frame}

\begin{frame}[fragile]{Decoupled Components}
\begin{itemize}
\item Using the read/valid handshake
\item Some flexibility on sending and receiving
\item Data is transferred when ready and valid are asserted
\item Chisel provides a standard bundle for this: \code{DecoupledIO}
\shortlist{../code/fifo_decoupled.txt}
\end{itemize}
\end{frame}

\begin{frame}[fragile]{Methods on \code{DecoupledIO}}
\begin{itemize}
\item \code{fire()} is true if and only if ready and valid are asserted
\item \code{enq(data)} sets data and sets valid to true (no check on ready)
\item \code{noenq()} deasserts valis.
\item \code{deq/nodeq} for the receiver side
\begin{chisel}
when(io.in.fire()) {
  // do the transfer
}
\end{chisel}
\item Sala convention: methods with no side effects, leaving off the empty parentheses
\begin{chisel}
when(io.in.fire) {
  // do the transfer
}
\end{chisel}
\end{itemize}
\end{frame}

\begin{frame}[fragile]{Arbiter}
\begin{itemize}
\item Several components have a request for a shared resource
\item Examples: bus, memory port, on-chip network ports
\item Needs to select which request gets granted
\item Different algorithms to do this
\item Simplest is priority, but unfair
\item Fairnis, and guaranteed progress, especially in real-time systems, is an issue
\item I have a fair arbiter in the Chisel book (10.6.2)
\end{itemize}
\end{frame}

\begin{frame}[fragile]{Chisel Arbiter}
\begin{itemize}
\item Uses \code{DecoupledIO} for inputs and the output
\begin{chisel}
val arb = Module(new Arbiter(UInt(), 2))
arb.io.in(0) <> producer0.io.out
arb.io.in(1) <> producer1.io.out
consumer.io.in <> arb.io.out
\end{chisel}
\item \code{Arbiter} is a simple priority based arbitration
\item \code{RRArbiter} Chosen in round robin
\begin{itemize}
\item No details on implementation
\item True, single cycle, round robin is combinational complex
\end{itemize}
\item \code{LockingRRArbiter} grant reserved for several clock cycles
\end{itemize}
\end{frame}

\begin{frame}[fragile]{Crossbar}
\begin{itemize}
\item N inputs are connected to M outputs
\item Needs M arbitration circuits, each of N inputs
\item Need an address for the destination
\item Scales only to a moderate N and M
\item Add pipelining for larger N and M
\item Switch to a Network-on-Chip
\end{itemize}
\end{frame}


\begin{frame}[fragile]{FIFO Queue}
\begin{figure}
  \centering
  \includegraphics[scale=\scale]{../figures/fifo}
  \caption{A writer, a FIFO buffer, and a reader.}
  \label{fig:fifo}
\end{figure}
\begin{itemize}
\item FIFO style queues between a sender and a receiver
\item The sender/write enqueues/writes into the Queue
\item The receiver/reader dequeues/reads from the Queue
\item Can \emph{even out} bursty traffic
\end{itemize}
\end{frame}

\begin{frame}[fragile]{Queues}
\begin{itemize}
\item Used when there is bursty traffic
\begin{itemize}
\item E.g., on a serial port
\item Write faster from CPU than serial can send
\item Receive data when CPU is busy with other stuff
\end{itemize}
\item But can fill up when the sender is continuously \emph{faster}
\item Showing implementation variations in the Chisel book
\begin{itemize}
\item Bubble FIFO (was proposed in lab 5)
\item double buffer, memory-based, ...
\end{itemize}
\end{itemize}
\end{frame}

\begin{frame}[fragile]{Bubble FIFO Example}
\begin{itemize}
\item FIFO interface
\shortlist{../code/fifo_io.txt}
\item Abstract base class (with some common code)
\shortlist{../code/fifo_abstract.txt}
\item \href{https://github.com/schoeberl/chisel-book/blob/master/src/main/scala/fifo/fifo.scala}{FIFO code}
\end{itemize}
\end{frame}

\begin{frame}[fragile]{Bubble FIFO}
\begin{itemize}
\item Simple, easy to understand
\item Uses minimal resources
\item However, each buffer stage has to toggle between empty and full
\item Maximum bandwidth is one word every two clock cycles
\item When full, needs N cycles for a restart
\begin{itemize}
\item The free element \emph{bubbles} towards the input
\end{itemize}
\item Better solutions
\begin{itemize}
\item Double buffer
\item Memory with read and write pointers
\end{itemize}
\end{itemize}
\end{frame}

\begin{frame}[fragile]{Use Inheritance}
\begin{itemize}
\item Defined an abstract base class for the FIFO and extend it
\begin{itemize}
\item Also use traits to inherit from several sources
\end{itemize}
\item Learn from software development to share code
\item We select the implementation by the type
\begin{chisel}
class DoubleBufferFifo[T <: Data](gen: T, depth: Int) extends Fifo(gen: T, depth: Int) {
...
class RegFifo[T <: Data](gen: T, depth: Int) extends Fifo(gen: T, depth: Int) {
...
class MemFifo[T <: Data](gen: T, depth: Int) extends Fifo(gen: T, depth: Int) {
...
\end{chisel}
\item Explore the different implementation details in Section 11.3 of the Chisel book

\end{itemize}
\end{frame}

\begin{frame}[fragile]{Chisel \code{Queue}}
\begin{itemize}
\item Chisel has a \code{Queue}
\item Interfaces are \code{DecoupledIO}s
\item Specify type and number of entries Queue(UInt(4.W), 8)
\item Optional arguments
\begin{itemize}
\item \code{pipe}: if full, allow concurrent enqueue and dequeue
\item \code{flow}: if empty, enqueued value available immediately for dequeue
\end{itemize}
\item What is the implementation? Resources? Throughput?
\end{itemize}
\begin{chisel}
val q = Module(new Queue(UInt(), 16))
q.io.enq <> producer.io.out
consumer.io.in <> q.io.deq
\end{chisel}
\end{frame}

\begin{frame}[fragile]{Connect a Network-on-Chip}
\begin{itemize}
\item n x n router in bi-thorus configuration
\item Each raouter has 4 + 1 ports: North, South, West, East, and local
\item Use a helper function \code{connect(...)}
\end{itemize}
\begin{chisel}
  def connect(r1: Int, p1: Int, r2: Int, p2: Int): Unit = {
    net(r1).io.ports(p1).in := net(r2).io.ports(p2).out
    net(r2).io.ports(p2).in := net(r1).io.ports(p1).out
  }

  for (i <- 0 until n) {
    for (j <- 0 until n) {
      val r = i * n + j
      connect(r, EAST, i * n + (j + 1) % n, WEST)
      connect(r, SOUTH, (i + 1) % n * n + j, NORTH)
    }
  }
\end{chisel}
\end{frame}

\begin{frame}[fragile]{Why Open Source?}
\begin{itemize}
\item Can improve your visibility
\item Add your GitHub repos to your CV
\item When hiring, I look up GitHub contributions
\item You can get help from the community
\item I got free translations of my Chisel book
\item Why not?
\begin{itemize}
\item You already use open-source tools and libraries
\item So return some to the community
\item Even in a company, if there is no patent
\item Oticon contributes to Zephyr
\end{itemize}
\end{itemize}
\end{frame}


\begin{frame}[fragile]{Open-Source Licensing}
\begin{itemize}
\item You own the copyright when creating code
\begin{itemize}
\item Hardware is different in DK, so it is blurry
\item Maybe your employer owns the copyright (not at DTU)
\end{itemize}
\item A license grants others to use and change your code
\begin{itemize}
\item Different licenses have different permissions and restrictions
\end{itemize}
\item When releasing, add a license to your repo
\begin{itemize}
\item Have it clearly visible (e.g., as LICENSE in the project root)
\item GitHub will detect it
\end{itemize}
\end{itemize}
\end{frame}


\begin{frame}[fragile]{License Types}
\begin{itemize}
\item BSD and MIT
\begin{itemize}
\item Commonly used, good for academic
\item Basically, use my stuff, but don't sue me
\item Can also be used in a commercial product (e.g., network stack)
\end{itemize}
\item GPL
\begin{itemize}
\item Commonly used in software
\item Copyleft means you have to make all changes available
\item Some companies have restrictions on using GPL software
\end{itemize}
\item Apache
\begin{itemize}
\item Similar to BDS and MIT
\item Includes explicit patent protection
\end{itemize}
\end{itemize}
\end{frame}


\begin{frame}[fragile]{Documentation}
\begin{itemize}
\item Summarize the purpose of your project
\item Instruct how to use it
\item List dependencies
\item Good documentation:
\begin{itemize}
\item May attract users
\item May even attract contributors
\item Forces you to rethink your project
\end{itemize}
\item Start early with the documentation
\end{itemize}
\end{frame}


\begin{frame}[fragile]{README.md}
\begin{itemize}
\item Well-suited for small to medium projects
\item Immediately visible on GitHub
\item Use markdown for simple formatting
\item Include figures, best in SVG
\item I will consider the README as part of the grade
\item Up to now, not much is going on
\begin{itemize}
\item Don't wait till the final week!
\end{itemize}
\end{itemize}
\end{frame}

\begin{frame}[fragile]{On Chat Tutor}
\begin{itemize}
\item AI trained on Chisel material
\item Are you using it?
\item \href{https://chattutor.dk/c/111/s/141/}{Chattutor}
\item Marius will come for feedback on Tuesday, November 25th
\end{itemize}
\end{frame}

\begin{frame}[fragile]{On the Project}
\begin{itemize}
\item Today is your presentation and demonstration
\begin{itemize}
\item Let us use reverse the order, e.g., FFT Core first
\end{itemize}
\item No lab
\item No report to be written
\item But I expect a very detailed README.md
\item Explain what it does, how to use
\end{itemize}
\end{frame}

\begin{frame}[fragile]{Next Week}
\begin{itemize}
\item Guest lecture by Hans Jakob on
\item Chisel in Research
\item Includes a lab exercise
\end{itemize}
\end{frame}


\begin{frame}[fragile]{Summary}
\begin{itemize}
\item Collect ideas for a more productive generator code
\item Did you find some Scala tricks that I did not present?
\begin{itemize}
\item You could add it to the course evaluation
\item Easier, add to our shared Google docs
\item Also, what you are missing
\end{itemize}
\item Use ready/valid interfaces (DecoupledIO)
\item Use queues, arbitration
\item Go open source!
\end{itemize}
\end{frame}

\end{document}

\begin{frame}[fragile]{XXX}
\begin{itemize}
\item xxx
\item xxx
\item xxx
\begin{itemize}
\item xxx
\end{itemize}
\item xxx
\begin{chisel}
yyy
\end{chisel}
\item xxx
\end{itemize}
\end{frame}

\begin{frame}[fragile]{Title}
\begin{itemize}
\item abc
\end{itemize}
\end{frame}
