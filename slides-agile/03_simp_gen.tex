\documentclass[xcolor={dvipsnames,table}]{beamer}
\usepackage{epsfig,graphicx}
\usepackage{palatino}
\usepackage{fancybox}
\usepackage{relsize}
\usepackage[procnames]{listings}
\usepackage{hyperref}
\usepackage{qtree} % needed?
\usepackage{booktabs}
\usepackage{dirtree}
\usepackage[normalem]{ulem}
\usepackage{tikz}
\usetikzlibrary{arrows.meta,positioning,calc,fit,shapes.geometric,decorations.pathreplacing}


% fatter TT font
\renewcommand*\ttdefault{txtt}
% another TT, suggested by Alex
% \usepackage{inconsolata}
% \usepackage[T1]{fontenc} % needed as well?


\newcommand{\scale}{0.7}

\newcommand{\todo}[1]{{\emph{TODO: #1}}}
\newcommand{\martin}[1]{{\color{blue} Martin: #1}}
\newcommand{\abcdef}[1]{{\color{red} Author2: #1}}

% uncomment following for final submission
%\renewcommand{\todo}[1]{}
%\renewcommand{\martin}[1]{}
%\renewcommand{\author2}[1]{}

\newcommand{\code}[1]{{\texttt{#1}}}

\hypersetup{
  linkcolor  = black,
%  citecolor  = blue,
  urlcolor   = blue,
  colorlinks = true,
}

\beamertemplatenavigationsymbolsempty
\setbeamertemplate{footline}[frame number]





\newif\ifbook
\input{../shared/chisel}

% TikZ for diagrams
\usepackage{tikz}
\usetikzlibrary{positioning, arrows.meta}

% Optional visual tweaks
\tikzset{
    >=Stealth,
    every node/.style={rounded corners=2pt}
}

\title{Simple Generators}
\author{Martin Schoeberl}
\date{\today}
\institute{Technical University of Denmark\\
Embedded Systems Engineering}

\begin{document}

\begin{frame}
\titlepage
\end{frame}

\begin{frame}[fragile]{TODO}
\begin{itemize}
\item Have the FBI story
\item Link the Scrum book
\item Repeat some Scala (and some Chisel)
\item Add stuff from xxx slides (e.g., pattern matching)
\item Do some initial generators
\item the magic Mux (generics)
\end{itemize}
\end{frame}

\begin{frame}[fragile]{Overview}
\begin{itemize}
\item A bit more Scala and Chisel
\item Functional programming for generators
\end{itemize}
\end{frame}

\begin{frame}[fragile]{Chisel}
\begin{itemize}
\item A hardware \emph{construction} language
\begin{itemize}
\item Constructing Hardware in a Scala Embedded Language
\item If it compiles, it is synthesizable hardware 
\item Say goodbye to your unintended latches
\end{itemize}
\item Chisel is not a high-level synthesis language
\item Single source for two targets
\begin{itemize}
\item Cycle accurate simulation (testing)
\item Verilog for synthesis
\end{itemize}
\item Embedded in Scala
\begin{itemize}
\item Full power of Scala available
\item We use Scala to write the generators
\end{itemize}
\item Developed at UC Berkeley
\end{itemize}
\end{frame}



\begin{frame}[fragile]{Chisel Example: 2-bit Counter}
\begin{verbatim}
class Counter extends Module {
  val io = IO(new Bundle {
    val out = Output(UInt(2.W))
  })
  val count = RegInit(0.U(2.W))
  count := count + 1.U
  io.out := count
}
\end{verbatim}
\end{frame}

\begin{frame}[fragile]{Example Test}
\begin{verbatim}
test(new Counter) { c =>
  c.io.out.expect(0.U)
  c.clock.step()
  c.io.out.expect(1.U)
  c.clock.step()
  c.io.out.expect(2.U)
}
\end{verbatim}
\end{frame}



\begin{frame}[fragile]{Chisel and Scala}
\begin{itemize}
\item Chisel is a library written in Scala
\begin{itemize}
\item Import the library with \code{import chisel3.\_}
\end{itemize}
\item Chisel code is Scala code
\item When it is run is \emph{generates} hardware
\begin{itemize}
\item Verilog for synthesis and simulation
\end{itemize}
\item Chisel is an embedded domain-specific language
\item Two languages in one can be a little bit confusing
\item We use Scala to program the hardware generators
\end{itemize}
\end{frame}






%\begin{frame}[fragile]{Generic Components}
%\begin{chisel}
%val c = Mux(cond, a, b)
%\end{chisel}
%\begin{itemize}
%\item This is a multiplexer
%\item Input can be any type
%\end{itemize}
%\end{frame}

\begin{frame}[fragile]{Functions}
\begin{itemize}
\item Circuits can be encapsulated in functions
\item Each \emph{function call} generates hardware
\item Simple functions can be a single line
\end{itemize}
\begin{chisel}
  def adder(v1: UInt, v2: UInt) = v1 + v2
  
  val add1 = adder(a, b)
  val add2 = adder(c, d)
\end{chisel}
\end{frame}

\begin{frame}[fragile]{Pure Functions}
\begin{itemize}
    \item A pure function:
    \begin{itemize}
        \item Always returns the same output for the same input
        \item Has no side effects
    \end{itemize}
\end{itemize}

\begin{verbatim}
def add(a: Int, b: Int): Int = a + b
\end{verbatim}

\begin{itemize}
    \item Combinational hardware modules are pure
    \item Side effects (state, IO, randomness) are controlled explicitly
\end{itemize}
\end{frame}


\begin{frame}[fragile]{Functional Abstraction}
\begin{chisel}
  def addSub(add: Bool, a: UInt, b: UInt) =
    Mux(add, a+b, a-b)

  val res = addSub(cond, a, b)
  
  def rising(d: Bool) = d && !RegNext(d)
\end{chisel}
\begin{itemize}
\item Functions for repeated pieces of logic
\item May contain state
\item Functions may return \emph{hardware}
\end{itemize}
\end{frame}

\begin{frame}[fragile]{The Counter as a Function}
\begin{itemize}
\item Longer functions in curly brackets
\item Last value is the return value
\end{itemize}
\begin{chisel}
def counter(n: UInt) = {
  
  val cntReg = RegInit(0.U(8.W))
  
  cntReg := cntReg + 1.U
  when(cntReg === n) {
    cntReg := 0.U
  }
  cntReg
}

val counter100 = counter(100.U)
\end{chisel}
\end{frame}


%\begin{frame}[fragile]{Functions}
%\begin{itemize}
%\item Example from Patmos execute stage
%\end{itemize}
%\begin{chisel}
%def alu(func: Bits, op1: UInt, op2: UInt): Bits = {
%  val result = UInt(width = DATA_WIDTH)
%  // some more lines...
%  switch(func) {
%    is(FUNC_ADD) { result := sum }
%    is(FUNC_SUB) { result := op1 - op2 }
%    is(FUNC_XOR) { result := (op1 ^ op2).toUInt }
%    // some more lines
%  }
%  result
%}
%\end{chisel}
%\end{frame}




\begin{frame}[fragile]{Use Functional Programming for Generators}
\shortlist{../code/fun_first.txt}
\shortlist{../code/fun_func_lit.txt}
\shortlist{../code/fun_reduce_tree.txt}
\begin{itemize}
\item This is a simple example
\item What about an arbitration tree with fair arbitration?
\end{itemize}
\end{frame}

\begin{frame}[fragile]{Functional Generation}
\begin{itemize}
\item Anonymous functions, called \emph{function literal}
\begin{chisel}
  (param) => function body
\end{chisel}
\item A function for a minimum search
\shortlist{../code/fun_min.txt}
\end{itemize}
\end{frame}

\begin{frame}[fragile]{Minimal Function with Index}
\begin{itemize}
\item Was the example for Tjark's heap sort
\shortlist{../code/fun_min2.txt}
\item We need an extra bundle to hold both values
\item A for loop is not so functional
\end{itemize}
\end{frame}

\begin{frame}[fragile]{We Can Use Tuples and zipWithIndex}
\begin{itemize}
\item \code{zipWithIndex} transforms the original sequence to a sequence of tuples
with second element is the index
\item Use \code{map} to translate from Scala \code{Int} to Chisel \code{UInt}
\item \code{reduce} does the minimum function, actually called 2 times (we could optimize this)
\shortlist{../code/fun_min3.txt}
\item The result is a Scala \code{Vector} and not a Chisel \code{Vec}
\end{itemize}
\end{frame}

\begin{frame}[fragile]{Solution with a Vec}
\begin{itemize}
\item This results in a Chisel \code{Vec}
\shortlist{../code/fun_min4.txt}
\item
\item We should add a \code{reduceTree} to the Scala sequence version (\code{TraversableOnce})
\end{itemize}
\end{frame}



\begin{frame}[fragile]{A Simple Tester}
\begin{itemize}
\item Just using \code{println} for manual inspection
\shortlist{../code/test_bench_simple.txt}
\end{itemize}
\end{frame}


\begin{frame}[fragile]{A Test}
\begin{itemize}
\item Poke values and \code{expect} some output
\shortlist{../code/test_bench.txt}
\end{itemize}
\end{frame}

%\begin{frame}[fragile]{Testing}
%\begin{itemize}
%\item Within Chisel with a tester (= Scala program)
%\item May include waveform generation
%\item peek and poke to read and set values
%\begin{itemize}
%\item Remember the BASIC days ;-)
%\end{itemize}
%\item printf in simulation on rising edge
%\begin{chisel}
%printf("Counting %x\n", r1)
%\end{chisel}
%\end{itemize}
%\end{frame}
%
%\begin{frame}[fragile]{Testing Example}
%\shortlist{../code/test_bench_simple.txt}
%\end{frame}


\begin{frame}[fragile]{Lab 2}
\begin{itemize}
\item Write a search for the maximum circuit (with \code{treeReduce()})
\item Optional: add the generation of the index of the maximum value
\item Emit Verilog with \code{emitVerilog(new Comp())}, so you can synthesize it at the Thursday session
\item Code is in \href{https://github.com/schoeberl/agile-hw/tree/main/lab2}{lab2}
\end{itemize}
\end{frame}

\begin{frame}[fragile]{Lecture 3}
\begin{itemize}
\item Hardware generators
\item Object-oriented hardware design
\end{itemize}
\end{frame}


\begin{frame}[fragile]{Scala List for Enumeration}
\begin{chisel}
  val empty :: full :: Nil = Enum(2)
\end{chisel}
\begin{itemize}
\item Can be used in wires and registers
\item Symbols for a state machine
\end{itemize}
\end{frame}

\begin{frame}[fragile]{Finite State Machine}
\begin{chisel}
  val empty :: full :: Nil = Enum(2)
  val stateReg = RegInit(empty)
  val dataReg = RegInit(0.U(size.W))

  when(stateReg === empty) {
    when(io.enq.write) {
      stateReg := full
      dataReg := io.enq.din
    }
  }.elsewhen(stateReg === full) {
    when(io.deq.read) {
      stateReg := empty
    }
  }
\end{chisel}
\begin{itemize}
\item A simple buffer for a bubble FIFO
\end{itemize}
\end{frame}

\begin{frame}[fragile]{Component Generation}
\begin{chisel}
val cores = new Array[Module](32)

for (j <- 0 until 32)
  cores(j) = Module(new CPU())
\end{chisel}
\begin{itemize}
\item Use Scala array to collect components
\item Generation with a Scala loop
\end{itemize}
\end{frame}

\begin{frame}[fragile]{Logic Generation}
\begin{itemize}
\item Read a file into a table
\begin{itemize}
\item E.g., to read in ROM content for a processor
\end{itemize}
\item Generate a truth table algorithmically
\begin{itemize}
\item E.g., generate binary to BCD translation
\end{itemize}
\item Use the full power of Scala
\end{itemize}
\begin{chisel}
val byteArray = Files.readAllBytes(Paths.get(fileName))
val arr = new Array[Bits](byteArray.length)
for (i <- 0 until byteArray.length) {
  arr(i) = Bits(byteArray(i), 8)
}
val rom = Vec[Bits](arr)
\end{chisel}
\end{frame}

\begin{frame}[fragile]{Parameterization}
\begin{chisel}
class ParamChannel(n: Int) extends Bundle {
  val data = Input(UInt(n.W))
  val ready = Output(Bool())
  val valid = Input(Bool())
}

val ch32 = new ParamChannel(32)
\end{chisel}
\begin{itemize}
\item Bundles and modules can be parametrized
\item Pass a parameter in the constructor
\end{itemize}

\end{frame}
\begin{frame}[fragile]{A Module with a Parameter}
\begin{chisel}
class ParamAdder(n: Int) extends Module {
  val io = IO(new Bundle {
    val a = Input(UInt(n.W))
    val b = Input(UInt(n.W))
    val result = Output(UInt(n.W))
  })

  val addVal = io.a + io.b
  io.result := addVal
}

val add8 = Module(new ParamAdder(8))
\end{chisel}
\begin{itemize}
\item Parameter can also be a Chisel type
\item Can also be a generic type:
\item \code{class Mod[T <: Bits](param: T) extends...}
\end{itemize}
\end{frame}

\begin{frame}[fragile]{Scala \code{for} Loop for Circuit Generation}
\begin{chisel}
val shiftReg = RegInit(0.U(8.W))

shiftReg(0) := inVal

for (i <- 1 until 8) {
  shiftReg(i) := shiftReg(i-1)
}
\end{chisel}
\begin{itemize}
\item \code{for} is Scala
\item This loop generates several connections
\item The connections are parallel hardware
\end{itemize}
\end{frame}

\begin{frame}[fragile]{Conditional Circuit Generation}
\begin{chisel}
class Base extends Module { val io = new Bundle() }
class VariantA extends Base { }
class VariantB extends Base { }

val m = if (useA) Module(new VariantA())
        else Module(new VariantB())
\end{chisel}
\begin{itemize}
\item \code{if} and \code{else} is Scala
\item \code{if} is an expression that returns a value
\begin{itemize}
\item Like ``\code{cond ? a : b;}'' in C and Java
\end{itemize}
\item This is not a hardware multiplexer
\item Decides which module to generate
\item Could even read an XML file for the configuration
\end{itemize}
\end{frame}

% use this for the 4 unit version
\begin{frame}[fragile]{Chisel has a Multiplexer}
\begin{figure}
  \includegraphics[scale=\scale]{../figures/mux}
\end{figure}
\shortlist{../code/mux.txt}
\begin{itemize}
\item So what?
\item Wait... What type is \code{a} and \code{b}?
\begin{itemize}
\item Can be any Chisel type!
\end{itemize}
\end{itemize}
\end{frame}

\begin{frame}[fragile]{Chisel has a Generic Multiplexer}
\begin{figure}
  \includegraphics[scale=\scale]{../figures/mux}
\end{figure}
\shortlist{../code/mux.txt}
\begin{itemize}
\item SW people may not be impressed
\item They have generics since Java 1.5 in 2004
\begin{itemize}
\item \code{List<Flowers> != List<Cars>}
\end{itemize}
\end{itemize}
\end{frame}


\begin{frame}[fragile]{Generics in Hardware Construction}
\begin{itemize}
\item Chisel supports generic classes with type parameters
\item Write hardware generators independent of concrete type
\item This is a multiplexer \emph{generator}
\end{itemize}
\shortlist{../code/param_func.txt}
\end{frame}

\begin{frame}[fragile]{Put Generics Into Use}
\begin{itemize}
\item Let us implement a generic FIFO
\item Use the generic ready/valid interface from Chisel
\end{itemize}
\shortlist{../code/fifo_decoupled.txt}
\end{frame}

\begin{frame}[fragile]{Define the FIFO Interface}
\shortlist{../code/fifo_io.txt}
\begin{itemize}
\item We need enqueueing and dequeueing ports
\item Note the \code{Flipped}
\begin{itemize}
\item It switches the direction of ports
\item No more double definitions of an interface
\end{itemize}
\end{itemize}
\end{frame}

\begin{frame}[fragile]{But What FIFO Implementation?}
\begin{itemize}
\item Bubble FIFO (good for low data rate)
\item Double buffer FIFO (fast restart)
\item FIFO with memory and pointers (for larger buffers)
\begin{itemize}
\item Using flip-flops
\item Using on-chip memory
\end{itemize}
\item And some more...
\end{itemize}
\begin{itemize}
\item This calls for object-oriented \sout{programming} \emph{hardware construction}
\end{itemize}
\end{frame}

\begin{frame}[fragile]{Abstract Base Class and Concrete Extension}
\shortlist{../code/fifo_abstract.txt}
\begin{itemize}
\item May contain common code
\item Extend by concrete classes
\end{itemize}
\begin{chisel}
class BubbleFifo[T <: Data](gen: T, depth: Int) extends Fifo(gen: T, depth: Int) {
\end{chisel}
\end{frame}



\begin{frame}[fragile]{Select a Concrete FIFO Implementation}
\begin{itemize}
\item Decide at hardware generation
\item Can use all Scala/Java power for the decision
\begin{itemize}
\item Connect to a web service, get \sout{Google} Alphabet stock price, and decide on which to use ;-)
\item For sure a silly idea, but you see what is possible...
\item Developers may find clever use of the Scala/Java power
\item We could present a GUI to the user to select from
\end{itemize}
\item We use XML files parsed at hardware generation time
\item End of TCL, Python,... generated hardware
\end{itemize}
\end{frame}


%
%
%
%\begin{frame}[fragile]{XXX}
%\begin{itemize}
%\item TODO: s4noc connection is part of the generator story
%\item
%\item
%\end{itemize}
%\end{frame}
%
%\begin{frame}[fragile]{XXX}
%\begin{itemize}
%\item
%\item
%\item
%\end{itemize}
%\end{frame}



\begin{frame}[fragile]{Combinational (Truth) Table Generation}
\begin{chisel}
val arr = new Array[Bits](length)
for (i <- 0 until length) {
  arr(i) = ...
}
val rom = Vec[Bits](arr)
\end{chisel}
\begin{itemize}
\item Generate a table in a Scala array
\item Use that array as input for a Chisel \code{Vec}
\item Generates a logic table at hardware construction time
\end{itemize}
\end{frame}

\begin{frame}[fragile]{Ideas for Runtime Table Generation}
\begin{itemize}
\item Assembler in Scala/Java generates the boot ROM
\item Table with a \code{sin} function
\item Binary to BCD conversion
\item Schedule table for a TDM-based network-on-chip
\item 
\item More ideas?
\end{itemize}
\end{frame}

\begin{frame}[fragile]{Test Driven Development (TDD)}
\begin{itemize}
\item Software development process
\begin{itemize}
\item Can we learn from SW development for HW design?
\end{itemize}
\item Writing the test first, then the implementation
\item Started with extreme programming
\begin{itemize}
\item Frequent releases
\item Accept change as part of the development
\end{itemize}
\item A path to \emph{Agile Hardware Development!}
\item Not used in its pure form
\begin{itemize}
\item Writing all those tests is simply considered too much work
\end{itemize}
\end{itemize}
\end{frame}

\begin{frame}[fragile]{Continuous Integration}
\begin{itemize}
\item Run your tests on each change
\item Do it also when using source control
\item GitHub Actions
\item I am doing it even for the Chisel book
\end{itemize}
\end{frame}

\begin{frame}[fragile]{Testing versus Debugging}
\begin{itemize}
\item Debugging is during code development
\item Waveform and println are easy tools for debugging
\item Debugging does not help for regression tests
\item Write small test cases for regression tests
\item Keeps your code base \emph{intact} when doing changes
\item Better confidence in changes not introducing new bugs
\end{itemize}
\end{frame}

\begin{frame}[fragile]{Lab 3}
\begin{itemize}
\item Design a component with a generic parameter
\item Emit Verilog with \code{emitVerilog(new Comp())}, you can synthesize it then on Thursday
\item Code is in \href{https://github.com/schoeberl/agile-hw/tree/main/lab3}{lab3}
\end{itemize}
\end{frame}

\begin{frame}[fragile]{Summary}
\begin{itemize}
\item Processors do not get much faster -- we need to design custom hardware
\item We need a modern language for hardware/systems design for efficient/fast development
\item Chisel builds on the power of object-oriented and functional Scala
\item We shall write hardware generators
\item We have a course at DTU on this topic this fall
\end{itemize}
\end{frame}


\end{document}

\begin{frame}[fragile]{Title}
\begin{itemize}
\item abc
\end{itemize}
\end{frame}
