\documentclass[xcolor=pdflatex,dvipsnames,table]{beamer}
\usepackage{epsfig,graphicx}
\usepackage{palatino}
\usepackage{fancybox}
\usepackage{relsize}
\usepackage[procnames]{listings}
\usepackage{hyperref}
\usepackage{qtree} % needed?
\usepackage{booktabs}
\usepackage{dirtree}
\usepackage[normalem]{ulem}


% fatter TT font
\renewcommand*\ttdefault{txtt}
% another TT, suggested by Alex
% \usepackage{inconsolata}
% \usepackage[T1]{fontenc} % needed as well?


\newcommand{\scale}{0.7}

\newcommand{\todo}[1]{{\emph{TODO: #1}}}
\newcommand{\martin}[1]{{\color{blue} Martin: #1}}
\newcommand{\abcdef}[1]{{\color{red} Author2: #1}}

% uncomment following for final submission
%\renewcommand{\todo}[1]{}
%\renewcommand{\martin}[1]{}
%\renewcommand{\author2}[1]{}

\newcommand{\code}[1]{{\texttt{#1}}}

\hypersetup{
  linkcolor  = black,
%  citecolor  = blue,
  urlcolor   = blue,
  colorlinks = true,
}

\beamertemplatenavigationsymbolsempty
\setbeamertemplate{footline}[frame number]





\newif\ifbook
% shared in slides and book

\lstdefinelanguage{chisel}{
  morekeywords={abstract,case,catch,class,def,%
    do,else,extends,false,final,finally,%
    for,if,implicit,import,match,mixin,%
    new,null,object,override,package,%
    private,protected,requires,return,sealed,%
    super,this,throw,trait,true,try,%
    type,val,var,while,with,yield},
  otherkeywords={=>,<-,<\%,<:,>:,\#,@},
  sensitive=true,
  morecomment=[l]{//},
  morecomment=[n]{/*}{*/},
  morestring=[b]",
  morestring=[b]',
  morestring=[b]"""
}

\usepackage{color}
\definecolor{dkgreen}{rgb}{0,0.6,0}
\definecolor{gray}{rgb}{0.5,0.5,0.5}
\definecolor{mauve}{rgb}{0.58,0,0.82}

% Default settings for code listings
\ifbook
\lstset{%frame=lines,
  language=chisel,
  aboveskip=3mm,
  belowskip=3mm,
  showstringspaces=false,
  columns=fixed, % basewidth=\mybasewidth,
  basicstyle={\small\ttfamily},
  numbers=none,
  numberstyle=\footnotesize,
  % identifierstyle=\color{red},
  breaklines=true,
  breakatwhitespace=true,
  procnamekeys={def, val, var, class, trait, object, extends},
  % procnamestyle=\ttfamily,
  tabsize=2,
  float
}
\else
\lstset{%frame=lines,
  language=chisel,
  aboveskip=3mm,
  belowskip=3mm,
  showstringspaces=false,
  columns=fixed, % basewidth=\mybasewidth,
  basicstyle={\small\ttfamily},
  numbers=none,
  numberstyle=\footnotesize\color{gray},
  % identifierstyle=\color{red},
  keywordstyle=\color{blue},
  commentstyle=\color{dkgreen},
  stringstyle=\color{mauve},
  breaklines=true,
  breakatwhitespace=true,
  procnamekeys={def, val, var, class, trait, object, extends},
  procnamestyle=\ttfamily\color{red},
  tabsize=2,
  float
}
\fi

\lstnewenvironment{chisel}[1][]
{\lstset{language=chisel,#1}}
{}

\newcommand{\shortlist}[1]{{\lstinputlisting[nolol]{#1}}}

\newcommand{\longlist}[3]{{\lstinputlisting[float, caption={#2}, label={#3}, frame=tb, captionpos=b]{#1}}}

\newcommand{\verylonglist}[3]{{\lstinputlisting[caption={#2}, label={#3}, frame=tb, captionpos=b]{#1}}}


\title{Testing and Continuous Integration}
\author{Emad Jacob Maroun}
\date{\today}
\institute{Technical University of Denmark\\
Embedded Systems Engineering}

\begin{document}

\begin{frame}
\titlepage
\end{frame}

\begin{frame}[fragile]{Outline}
\begin{itemize}
	 
	What is testing for
	Manual vs Automatic tests
	What levels of tests exist
		* Unit test
		* Integration/system tests (discover bugs)
		* Release testing (By outside people, makes sure things follow the requirements)
		* Performance testing
		* User testing
		
	TDD:
		* Flow: 1) Identify functionality
				2) Write test
				3) run tests (should fail)
				4) Implement functionality and refactor
				5) run tests
				6) Either succeed and go to 1) or fail and repeat 4)
				
		* Regression tests: All previous tests
		* Can be used in unit tests and integration tests.
		
	Property Based testing:
		* Many applications have infinite case space => cannot test all cases
		* PBT tries to uncover classes of bugs through randomized data generation
		* Strategy:
			- Identify a property
			- Define input data generator (only valid inputs)
			- Define golden model
			- Run many times
		* When a bug is found, a unit/integration test (not property based) excercising that specific case is created.
	
	CI:
	
		* Tests are only useful if they are run.
		* Problem: 
			- People are forgetful, have different machines, do not have time, are lazy
			- As projects grow, making sure everyone is running all tests becomes impossible.
			- Tests may also become so complex and time-consuming that it is impractical to ask everyone to run them all
			- Continuous integration solves the problem by automatic the running of tests
			
		* Solution:
			- Have a server continuously run tests
			- Pull requests ensure all tests succeed before code is merged
		
		* Benefits:
			- Developers do not need to run all tests locally
			- Can be repeated on all supported platforms
			- Regressions are caught before being merged
			- Master branch is always in a working state
			
		* Misc:
			- Can run performance tests regularly
			- Can also help with deployment
			- Useful for single-person projects
			
	Systolic Array Priority List:
		
		* Sorting structure
		* Push key and payload at the front
		* Command propagates up the array, updating the ordering
		* Pop smallest key and payload once every 2 cycles
		* Used in e.g. network routers
		* Low latency but expensive
		* Easily configurable
		
		Commands:
			* IDLE: Do nothing
					
			* PUSH: insert key/value pair into the array

			* POP: extract a pair from the array
			
			* Push/Pop: Extract pair from array while push new pair to array
			
		* Should be configurable in key size, value size, and array length			
			
		Extension:
			Values must be tracked in groups. 
			each key may have multiple values
			Each push may have varying number of values
			newly pushed values must merge with previously pushed ones.
	
			Commands:
				Push(0): do nothing
				Pop(0): Pop lowest key/pairs
				Push(x): Push x values with given key
				Pop (x): Pop lowest key/pairs, push x values with given key
				
		* Should be configurable in max pair lengths
		
		* How should this be tested:
			* Unit tests of array elements
			* Integration tests of whole array
			* Property based tests:
				- If you push a value, you can eventually pop it back out
				- If you push a set of values, you will pop them sorted
				- Idling saves elements
				- Extension: Successive elements of the same key come out in order

\end{itemize}
\end{frame}

\begin{frame}[fragile]{Agile Manifesto Focus}

\begin{centering}
	
	Individuals and \onslide*<1>{interactions}\onslide*<2->{\textbf{interactions}}  over processes and tools
	\vspace{1em}
	
	\onslide*<1>{Working software}\onslide*<2->{\textbf{Working}  \sout{software} \textbf{hardware}} over comprehensive documentation
	\vspace{1em}
	
	Customer collaboration over contract negotiation
	\vspace{1em}
	
	\onslide*<1>{Responding to change}\onslide*<2->{\textbf{Responding to change}} over following a plan \footnote{\href{https://agilemanifesto.org/}{https://agilemanifesto.org/}}

\end{centering}

\end{frame}





\end{document}

\begin{frame}[fragile]{Title}
\begin{itemize}
\item abc
\end{itemize}
\end{frame}
