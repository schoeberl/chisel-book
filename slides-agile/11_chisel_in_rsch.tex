\documentclass[xcolor=pdflatex,dvipsnames,table]{beamer}
\usepackage{epsfig,graphicx}
\usepackage{palatino}
\usepackage{fancybox}
\usepackage{relsize}
\usepackage[procnames]{listings}
\usepackage{hyperref}
\usepackage{qtree} % needed?
\usepackage{booktabs}
\usepackage{dirtree}
\usepackage[normalem]{ulem}


% fatter TT font
\renewcommand*\ttdefault{txtt}
% another TT, suggested by Alex
% \usepackage{inconsolata}
% \usepackage[T1]{fontenc} % needed as well?


\newcommand{\scale}{0.7}

\newcommand{\todo}[1]{{\emph{TODO: #1}}}
\newcommand{\martin}[1]{{\color{blue} Martin: #1}}
\newcommand{\abcdef}[1]{{\color{red} Author2: #1}}

% uncomment following for final submission
%\renewcommand{\todo}[1]{}
%\renewcommand{\martin}[1]{}
%\renewcommand{\author2}[1]{}

\newcommand{\code}[1]{{\texttt{#1}}}

\hypersetup{
  linkcolor  = black,
%  citecolor  = blue,
  urlcolor   = blue,
  colorlinks = true,
}

\beamertemplatenavigationsymbolsempty
\setbeamertemplate{footline}[frame number]





\newif\ifbook
% shared in slides and book

\lstdefinelanguage{chisel}{
  morekeywords={abstract,case,catch,class,def,%
    do,else,extends,false,final,finally,%
    for,if,implicit,import,match,mixin,%
    new,null,object,override,package,%
    private,protected,requires,return,sealed,%
    super,this,throw,trait,true,try,%
    type,val,var,while,with,yield},
  otherkeywords={=>,<-,<\%,<:,>:,\#,@},
  sensitive=true,
  morecomment=[l]{//},
  morecomment=[n]{/*}{*/},
  morestring=[b]",
  morestring=[b]',
  morestring=[b]"""
}

\usepackage{color}
\definecolor{dkgreen}{rgb}{0,0.6,0}
\definecolor{gray}{rgb}{0.5,0.5,0.5}
\definecolor{mauve}{rgb}{0.58,0,0.82}

% Default settings for code listings
\ifbook
\lstset{%frame=lines,
  language=chisel,
  aboveskip=3mm,
  belowskip=3mm,
  showstringspaces=false,
  columns=fixed, % basewidth=\mybasewidth,
  basicstyle={\small\ttfamily},
  numbers=none,
  numberstyle=\footnotesize,
  % identifierstyle=\color{red},
  breaklines=true,
  breakatwhitespace=true,
  procnamekeys={def, val, var, class, trait, object, extends},
  % procnamestyle=\ttfamily,
  tabsize=2,
  float
}
\else
\lstset{%frame=lines,
  language=chisel,
  aboveskip=3mm,
  belowskip=3mm,
  showstringspaces=false,
  columns=fixed, % basewidth=\mybasewidth,
  basicstyle={\small\ttfamily},
  numbers=none,
  numberstyle=\footnotesize\color{gray},
  % identifierstyle=\color{red},
  keywordstyle=\color{blue},
  commentstyle=\color{dkgreen},
  stringstyle=\color{mauve},
  breaklines=true,
  breakatwhitespace=true,
  procnamekeys={def, val, var, class, trait, object, extends},
  procnamestyle=\ttfamily\color{red},
  tabsize=2,
  float
}
\fi

\lstnewenvironment{chisel}[1][]
{\lstset{language=chisel,#1}}
{}

\newcommand{\shortlist}[1]{{\lstinputlisting[nolol]{#1}}}

\newcommand{\longlist}[3]{{\lstinputlisting[float, caption={#2}, label={#3}, frame=tb, captionpos=b]{#1}}}

\newcommand{\verylonglist}[3]{{\lstinputlisting[caption={#2}, label={#3}, frame=tb, captionpos=b]{#1}}}


\title{Chisel in Research}
\author{Hans Jakob Damsgaard}
\date{18 November 2025}
\institute{Teledyne RESON}

\definecolor{dkgreen}{rgb}{0,0.6,0}
\definecolor{gray}{rgb}{0.5,0.5,0.5}
\definecolor{mauve}{rgb}{0.58,0,0.82}

\lstdefinestyle{scstyle}{
  frame=tb,
  language=scala,
  aboveskip=3mm,
  belowskip=3mm,
  showstringspaces=false,
  columns=flexible,
  basicstyle={\small\ttfamily},
  numbers=none,
  numberstyle=\tiny\color{gray},
  keywordstyle=\color{blue},
  commentstyle=\color{dkgreen},
  stringstyle=\color{mauve},
  frame=none,
  breaklines=true,
  breakatwhitespace=true,
  tabsize=3,
}

\usepackage{subcaption}
\usepackage{tabularx}

\AtBeginSection[]{
  \begin{frame}
  \vfill
  \centering
  \begin{beamercolorbox}[sep=8pt,center,shadow=true,rounded=true]{title}
    \usebeamerfont{title}\NoHyper\insertsectionhead\endNoHyper\par%
  \end{beamercolorbox}
  \vfill
  \end{frame}
}

\begin{document}

\begin{frame}
\titlepage
\end{frame}

\begin{frame}{Outline}
\begin{itemize}
    \item \textbf{Researching Approximate Computing with Chisel}
    \begin{itemize}
        \item What is approximate computing?
        \item How has Chisel accelerated my research?
        \item Reflections on code patterns and strategies
    \end{itemize}
    \item \textbf{Life as a PhD Student}
    \item \textbf{Life after PhD Studies}
    \item \textbf{Introduction to Today's Lab} with Tjark
\end{itemize}
\end{frame}

\section{Researching Approximate Computing\\with Chisel}
\begin{frame}[fragile]{What is Approximate Computing?}
\framesubtitle{The Basics}
\begin{figure}
    \centering
    \includegraphics[width=.8\textwidth]{images/AxC_energy_50p.pdf}
\end{figure}
\vspace{1em}
\begin{centering}
    ``\textit{Exploit \textbf{error tolerance} to trade off\\\textbf{computational accuracy} for \textbf{disproportionate\\savings} in energy, delay, circuit area, etv.}''

\end{centering}
\end{frame}

\begin{frame}{What is Approximate Computing?}
\framesubtitle{The Applications}
\begin{itemize}
    \onslide<1->{
        \item Computationally intensive algorithms
        \item Negligible output quality degradation
    }
    \onslide<2->{
        \begin{itemize}
            \item Large data aggregation (no golden result)
            \item Audible/visual results consumed by humans
        \end{itemize}
    }
\end{itemize}
\vspace{2em}
\begin{figure}
    \centering
    \onslide<3->{
        \begin{subfigure}{.25\textwidth}
            \centering
            \includegraphics[width=\textwidth]{images/media.pdf}
        \end{subfigure}
    }
    \hspace{1em}
    \onslide<4->{
        \begin{subfigure}{.25\textwidth}
            \centering
            \includegraphics[width=\textwidth]{images/neural-network.pdf}
        \end{subfigure}
    }
    \hspace{1em}
    \onslide<5->{
        \begin{subfigure}{.25\textwidth}
            \centering
            \includegraphics[width=\textwidth]{images/signal.pdf}
        \end{subfigure}
    }
\end{figure}
\end{frame}

\begin{frame}{How do We Approximate?}
\framesubtitle{The Hardware Techniques}
\begin{itemize}
    \onslide<1->{
        \item Circuit-level approximation
        \begin{itemize}
            \item Voltage over-scaling or near-threshold computing
            \item Data path truncation or inexact synthesis
        \end{itemize}
    }
    \vspace{1em}
    \onslide<2->{
        \item Stochastic computing
    }
    \vspace{1em}
    \onslide<3->{
        \item Inexact arithmetic
        \begin{itemize}
            \item Adders, multipliers, other operators
            \item Iterative algorithms and error compensation
        \end{itemize}
    }
\end{itemize}
\end{frame}

\begin{frame}{How do We Approximate?}
\framesubtitle{Focus on Arithmetic}
\onslide<1->{
    \begin{figure}
        \centering
        \includegraphics[width=\textwidth]{images/Inexact_arith.pdf}
    \end{figure}
}
\vspace{2em}
\onslide<2->{
    \begin{centering}
        Selectively enabled approximation?

    \end{centering}
}
\vspace{-2em}
\end{frame}

\begin{frame}{Error Characteristics}
\framesubtitle{Adaptive OFLOCA}
\begin{figure}
    \centering
    \includegraphics[width=\textwidth]{images/AdaptiveOFLOCA.pdf}
\end{figure}
\end{frame}

\begin{frame}{Error Characteristics}
\framesubtitle{Adaptive Radix-2 Multiplier}
\begin{figure}
    \centering
    \includegraphics[width=\textwidth]{images/AdaptiveRadix2Multiplier.pdf}
\end{figure}
\end{frame}

\begin{frame}{Error Characteristics}
\framesubtitle{Distributions}
\begin{figure}
    \centering
    \begin{subfigure}{.45\textwidth}
        \centering
        \includegraphics[width=\textwidth]{images/hist_lsesa1.pdf}
        \caption{LSESA1}
    \end{subfigure}
    \hspace{2em}
    \begin{subfigure}{.45\textwidth}
        \centering
        \includegraphics[width=\textwidth]{images/hist_gear.pdf}
        \caption{GeAr}
    \end{subfigure}
    \\
    \begin{subfigure}{.45\textwidth}
        \centering
        \includegraphics[width=\textwidth]{images/hist_axppa.pdf}
        \caption{AxPPA}
    \end{subfigure}
    \hspace{2em}
    \begin{subfigure}{.45\textwidth}
        \centering
        \includegraphics[width=\textwidth]{images/hist_aofloca.pdf}
        \caption{Adaptive OFLOCA}
    \end{subfigure}
\end{figure}
\end{frame}

\begin{frame}{Entering \code{approx}}
\begin{itemize}
    \onslide<1->{
        \item Open-source library of (in)exact arithmetic units\footnote{Available on \href{https://github.com/aproposorg/approx}{GitHub}. Histograms generated with \href{https://github.com/aproposorg/emixa}{emixa}.\\\quad\;\; High-level metrics may be generated with \href{https://github.com/chiselverify/chiselverify}{chiselverify.approximation}.}
    }
    \vspace{1em}
    \onslide<2->{
        \item Mostly adders and multipliers
        \begin{itemize}
            \item Composable full adders and approximate adders
            \item Radix-2 and radix-4 multipliers
            \item Custom compressor tree generator
            \item \textit{Recursive} multipliers
            \item Various (parallel) accumulators
        \end{itemize}
    }
    \vspace{.5em}
    \onslide<3->{
        \item Also serves as my personal Chisel playground
    }
\end{itemize}
\end{frame}

\begin{frame}[fragile]{Design Strategy of \code{approx}}
\begin{tabular*}{\textwidth}[t]{p{.4\textwidth}@{\extracolsep{\fill}}p{.58\textwidth}}
    \begin{itemize}
        \onslide<1->{
            \item Common interfaces\\(in \code{package.scala})
        }
        \vspace{1em}
        \onslide<2->{
            \item Functional first\\(over imperative)
        }
        \vspace{1em}
        \onslide<3->{
            \item Immutability rules\\(for sanity)
        }
    \end{itemize}
    &
    \onslide*<1>{
        \begin{figure}
            \centering
            \includegraphics[width=.3\textwidth]{images/package_scala.pdf}
        \end{figure}
    }
    \onslide*<2>{
        \begin{figure}
            \centering
            \includegraphics[width=.45\textwidth]{images/functional_first.pdf}
        \end{figure}
    }
    \onslide*<3->{
        \begin{figure}
            \centering
            \includegraphics[width=.56\textwidth]{images/immutability_rules.pdf}
        \end{figure}
    }
    \\
\end{tabular*}
\end{frame}

\begin{frame}{Making Use of \code{approx}}
\framesubtitle{Returning to Applications}
\onslide<1->{
    \begin{figure}
        \centering
        \begin{subfigure}{.25\textwidth}
            \centering
            \includegraphics[width=\textwidth]{images/media.pdf}
        \end{subfigure}
        \hspace{1em}
        \begin{subfigure}{.25\textwidth}
            \centering
            \includegraphics[width=\textwidth]{images/neural-network.pdf}
        \end{subfigure}
        \hspace{1em}
        \begin{subfigure}{.25\textwidth}
            \centering
            \includegraphics[width=\textwidth]{images/signal.pdf}
        \end{subfigure}
    \end{figure}
}
\vspace{-1em}
\onslide*<2>{
    \begin{figure}
        \centering
        \includegraphics[width=.75\textwidth]{images/diverse_clean.pdf}
    \end{figure}
}
\onslide*<3>{
    \begin{figure}
        \centering
        \includegraphics[width=.75\textwidth]{images/diverse_prohibited.pdf}
    \end{figure}
    \begin{centering}
        Individual accelerators infeasible ...

    \end{centering}
}
\onslide*<4->{
    \begin{figure}
        \centering
        \includegraphics[width=.75\textwidth]{images/diverse_reconf.pdf}
    \end{figure}
    \begin{centering}
        ... let's use a reconfigurable one!

    \end{centering}
}
\end{frame}

\begin{frame}{What is a CGRA?}
\framesubtitle{The Basics}
\begin{itemize}
    \onslide<1->{
        \item \textit{Coarse-Grained Reconfigurable Array}
        \begin{itemize}
            \item (Heterogeneous) mesh of \textit{Processing Elements} (PEs)
            \item Adaptive data path and routing
            \item \textit{Word}-level operations (compared with FPGAs)
        \end{itemize}
    }
    \vspace{1em}
    \onslide<2->{
        \item Meant for computationally intensive applications
        \begin{itemize}
            \item Small, repeated kernel executions
            \item Spatio-temporal dataflow (from/to memories)
            \item Balancing performance, flexibility, and energy efficiency
        \end{itemize}
    }
\end{itemize}
\end{frame}

\begin{frame}{What is a CGRA?}
\framesubtitle{Example Architecture}
\onslide<1->{
    \begin{figure}
        \centering
        \includegraphics[width=.8\textwidth]{images/CGRA.pdf}
    \end{figure}
}
\onslide<2->{
    \begin{centering}
        Loads of architectural options!

    \end{centering}
}
\end{frame}

\begin{frame}{What is a CGRA?}
\framesubtitle{Architectural Diversity}
\begin{figure}
    \centering
    \begin{subfigure}{.45\textwidth}
        \centering
        \includegraphics[height=2.8cm]{images/PE_ADRES_v3.pdf}
        \caption{ADRES}
    \end{subfigure}
    \hspace{2em}
    \begin{subfigure}{.45\textwidth}
        \centering
        \includegraphics[height=2.8cm]{images/PE_CMA_v2.pdf}
        \caption{CMA}
    \end{subfigure}
    \\
    \begin{subfigure}{.45\textwidth}
        \centering
        \includegraphics[height=2.8cm]{images/PE_HyCUBE_v3.pdf}
        \caption{HyCUBE}
    \end{subfigure}
    \hspace{2em}
    \begin{subfigure}{.45\textwidth}
        \centering
        \includegraphics[height=2.8cm]{images/PE_RF-CGRA_v2.pdf}
        \caption{RF-CGRA}
    \end{subfigure}
\end{figure}
\end{frame}

\begin{frame}[fragile]{Open-Source CGRA Tools}
\framesubtitle{Supporting Architectural Diversity}
\begin{table}[t]
    \centering
    \footnotesize
    \setlength\tabcolsep{4pt}
    \begin{tabular}{@{}llcccc@{}}
    \toprule
    \textbf{Framework} & \textbf{Lang.} & \textbf{Generic?} & \textbf{Flex. ADL?} & \textbf{Mapping?} & \textbf{Working?} \\ \midrule
    \href{https://ieeexplore.ieee.org/document/7995277/}{CGRA-ME} & C++ & \checkmark & \checkmark & $\sim$ & \checkmark\textsuperscript{\textit{a}} \\
    \href{https://past.date-conference.com/system/files/file/date18/ubooth/121371.pdf}{CCF} & C++ & --- & --- & \checkmark & \checkmark \\
    \href{https://woset-workshop.github.io/PDFs/2020/a22.pdf}{Pillars}\footnote{Pillars also uses Chisel!} & Scala & \checkmark & --- & $\sim$ & \checkmark \\
    \href{https://ieeexplore.ieee.org/document/9283606/}{OpenCGRA} & Python & --- & \checkmark & \checkmark & \checkmark \\
    \href{https://ieeexplore.ieee.org/document/9609928}{FastCGRA} & Python & --- & \checkmark & \checkmark & \checkmark\textsuperscript{\textit{b}} \\
    \href{https://dl.acm.org/doi/10.1145/3503222.3507772}{REVAMP} & C++ & \checkmark & --- & $\sim$ & \checkmark \\
    \href{https://woset-workshop.github.io/PDFs/2022/12-Wijerathne-paper.pdf}{Morpher} & C++ & \checkmark & --- & \checkmark & \checkmark\textsuperscript{\textit{c}} \\
    \href{https://ieeexplore.ieee.org/document/10374147}{HETA} & C++ & --- & \checkmark & \checkmark & \checkmark\textsuperscript{\textit{c}} \\ \midrule
    CGRAgen & Scala & \checkmark & \checkmark & \checkmark & \checkmark \\ \bottomrule
    \end{tabular}
\end{table}
\end{frame}

\begin{frame}{What is CGRAgen?}
\framesubtitle{Overview}
\begin{figure}
    \centering
    \includegraphics[width=\textwidth]{images/CGRAgen_flow_red.pdf}
\end{figure}
\end{frame}

\begin{frame}[fragile]{Architecture Description}
\begin{tabular*}{\textwidth}[t]{p{.4\textwidth}@{\extracolsep{\fill}}p{.58\textwidth}}
    \vspace{2em}
    \begin{itemize}
        \onslide<1->{
            \item Simple ADL\\(based on XML)
            \begin{itemize}
                \item CGRA-ME-inspired
                \item Intg. \code{scala-xml}
            \end{itemize}
        }
        \vspace{1em}
        \onslide<2->{
            \item Composable module\\templates
        }
        \vspace{1em}
        \onslide<3->{
            \item Five hardware\\primitives
        }
    \end{itemize}
    &
    \onslide<1->{
        \begin{figure}
            \includegraphics[width=.55\textwidth]{images/cgra_adl.pdf}
        \end{figure}
    }
    \\
\end{tabular*}
\end{frame}

\begin{frame}[fragile]{Application Mapping}
\framesubtitle{Intermediate Modeling}
\begin{tabular*}{\textwidth}[t]{p{.5\textwidth}@{\extracolsep{\fill}}p{.49\textwidth}}
    \begin{itemize}
        \onslide<1->{
            \item Application kernels as\\\textit{Dataflow Graphs}
            \begin{itemize}
                \item Nodes are operations
                \item Edges are data dependencies
            \end{itemize}
        }
        \vspace{1em}
        \onslide<2->{
            \item CGRAs as \textit{Modulo Resource Routing Graphs}
            \begin{itemize}
                \item Differ with target \textit{II}
                \item Register example for\\$II=1$ and $II=2$
                \item ALU example for $II=2$
            \end{itemize}
        }
    \end{itemize}
    &
    \vspace{1em}
    \onslide*<2>{
        \begin{figure}
            \centering
            \begin{subfigure}{.33\textwidth}
                \centering
                \includegraphics[width=\textwidth]{images/RegMRRG1.pdf}
            \end{subfigure}
            \\
            \begin{subfigure}{.33\textwidth}
                \centering
                \includegraphics[width=\textwidth]{images/RegMRRG2.pdf}
            \end{subfigure}
        \end{figure}
    }
    \onslide*<3->{
        \begin{figure}
            \centering
            \includegraphics[width=.33\textwidth]{images/ALUMRRG2.pdf}
        \end{figure}
    }
    \\
\end{tabular*}
\end{frame}

\begin{frame}{Application Mapping}
\framesubtitle{Mapping and Approximation Algorithms}
\begin{itemize}
    \onslide<1->{
        \item Topologically sort the DFG and find components
    }
    \vspace{1em}
    \onslide<2->{
        \item Recursively build a mapping
        \begin{enumerate}
            \item Select an unmapped node
            \item Compute all placement and routing options
            \item Try (a subset of) these options
            \item Unmap predecessors and successors if unsuccessful
        \end{enumerate}
    }
    \vspace{1em}
    \onslide<3->{
        \item Aproximate through simulation
        \begin{itemize}
            \item Simulate arithmetic to measure (and cache) variance
            \item ``Execute'' DFG to estimate \textit{error sensitivity}
            \item Propagate variance and select approximation modes
        \end{itemize}
    }
\end{itemize}
\vspace{1em}
\onslide<4->{
    \begin{centering}
        Difficult to get this right; can deadlock on high utilization!\\(The problem is $NP$-hard...)

    \end{centering}
}
\end{frame}

\begin{frame}[fragile]{Model to Hardware}
\framesubtitle{\code{AbstractModule}}
\begin{tabular*}{\textwidth}[t]{p{.44\textwidth}@{\extracolsep{\fill}}p{.54\textwidth}}
    \begin{itemize}
        \item Modeling hardware
        \begin{itemize}
            \item Name and type
            \item Port (names, types, and directions)
            \item Sub-modules (names and types)
            \item Connections (port-to-port)
        \end{itemize}
        \vspace{1em}
        \item Configuration cells
    \end{itemize}
    &
    \onslide<2->{
        \begin{figure}
            \centering
            \includegraphics[width=.52\textwidth]{images/HWGen.pdf}
        \end{figure}
    }
    \\
\end{tabular*}
\end{frame}

\begin{frame}{Model to Hardware}
\framesubtitle{Generation Procedure}
\begin{tabular*}{\textwidth}[t]{p{.44\textwidth}@{\extracolsep{\fill}}p{.54\textwidth}}
    \begin{itemize}
        \item Bottom-up construction\\(\code{Module} class body)
        \begin{enumerate}
            \item Name the module
            \item Recursively create sub-modules
            \item Create data ports
            \item Connect ports and sub-modules
            \item If top, create control hardware; else, pass through
        \end{enumerate}
    \end{itemize}
    &
    \begin{figure}
        \centering
        \includegraphics[width=.52\textwidth]{images/HWGen.pdf}
    \end{figure}
    \\
\end{tabular*}
\end{frame}

\begin{frame}[fragile]{Model to Hardware}
\framesubtitle{Primitives}
\begin{tabular*}{\textwidth}[t]{p{.44\textwidth}@{\extracolsep{\fill}}p{.54\textwidth}}
    \begin{itemize}
        \onslide<1->{
            \item Pre-defined, usually inlined logic
            \begin{itemize}
                \item Storage and routing
                \item Arithmetic and logic
            \end{itemize}
        }
        \vspace{1em}
        \onslide<2->{
            \item Top-level IO primitives
            \begin{itemize}
                \item Inputs and outputs in hierarchy
                \item Bottom-up construction clashes?!
            \end{itemize}
        }
    \end{itemize}
    &
    \begin{figure}
        \centering
        \includegraphics[width=.52\textwidth]{images/HWGen.pdf}
    \end{figure}
    \\
\end{tabular*}
\end{frame}

\begin{frame}{Model to Hardware}
\framesubtitle{Handling IO Propagation}
\begin{itemize}
    \onslide<1->{
        \item Idea: make IO records \code{lazy}
        \begin{itemize}
            \item Collect IO in mutable \code{HashMap}s
            \item Elaborate when ready; no early referencing
        \end{itemize}
    }
    \vspace{1em}
    \onslide<2->{
        \item No traditional HDL supports this!
    }
    \vspace{1em}
    \onslide<3->{
        \item Flow must be carefully executed
        \begin{itemize}
            \item Inheritance $\Rightarrow$ parent constructor first
            \item Two methods \code{build} and \code{buildConfig}
            \item Primitives override \code{build}
        \end{itemize}
    }
\end{itemize}
\end{frame}

\begin{frame}{Model to Hardware}
\framesubtitle{Handling IO Propagation}
\begin{figure}
    \centering
    \includegraphics[width=.9\textwidth]{images/port_record.pdf}
\end{figure}
\begin{figure}
    \centering
    \includegraphics[width=.9\textwidth]{images/port_maps.pdf}
\end{figure}
\end{frame}

\begin{frame}{Model to Hardware}
\framesubtitle{Issues and Solutions}
\begin{itemize}
    \onslide<1->{
        \item Chisel uses context-based signal naming
    }
    \onslide<2->{
        \begin{itemize}
            \item Generated module IO should match specification
            \item Naming-related API frequently used
            \item \code{noPrefix}, \code{desiredName}, \code{suggestName}, and \code{forceName}
        \end{itemize}
    }
    \vspace{1em}
    \onslide<3->{
        \item Deduplication struggles with large-enough designs
    }
    \onslide<4->{
        \begin{itemize}
            \item Long generation time due to repeated elaboration
            \item Avoid re-elaboration with Chisel \href{https://www.chisel-lang.org/docs/cookbooks/hierarchy}{Hierarchy}
            \item All \code{Buildable}s are \code{@instantiable} with \code{@public} IO
            \item Maintain map of \code{Definition}s and create \code{Instance}s
        \end{itemize}
    }
\end{itemize}
\end{frame}

\begin{frame}{Does it Work?}
\onslide<1->{
    \begin{figure}
        \centering
        \includegraphics[width=\textwidth]{images/DFG_powers.pdf}
    \end{figure}
}
\vspace{-1em}
\onslide<2->{
    \begin{centering}
        Seems so - up to 15\% power savings\\(mind you, on an FPGA)

    \end{centering}
}
\end{frame}

\begin{frame}{Personal Advertisement}
\framesubtitle{Ideas and Plans for \code{approx} (curr. Chisel 7.0.0-RC1)}
\begin{enumerate}
    \item Extend library collection
    \vspace{1em}
    \item Build floating-point sub-package
    \begin{itemize}
        \item IEEE 754-style reference designs
        \item Approximated adders/multipliers
    \end{itemize}
    \vspace{1em}
    \item Automate design space exploration
    \begin{itemize}
        \item Scripts for ASIC/FPGA area/delay analyses
        \item Flow to establish/validate error bounds\footnote{Attempted with \href{https://github.com/chiselverify/chiselverify}{chiselverify.approximation} but should be formal.}
    \end{itemize}
    \vspace{1em}
    \item Use the library for more publications
    \begin{itemize}
        \item Novel arithmetic unit designs
        \item Reconfigurable or application-specific accelerators
    \end{itemize}
\end{enumerate}
\end{frame}

\begin{frame}{Personal Advertisement}
\framesubtitle{Ideas and Plans for \code{CGRAgen} (curr. Chisel 3.5.4)}
\begin{enumerate}
    \item Improve design flexibility and performance
    \begin{itemize}
        \item Pipelining and PE-to-PE handshaking interfaces \hyperref[sfig:handshaking]{(a)}
        \item CPU-to-CGRA and CGRA-to-memory interfaces
    \end{itemize}
    \vspace{.5em}
    \item Implement advanced CGRA features
    \begin{itemize}
        \item Runtime context switching
        \item Conditional execution through predication \hyperref[sfig:predication]{(b)}
    \end{itemize}
    \vspace{.5em}
    \item Automate verification
    \begin{itemize}
        \item Flexible test and script generation with mapping integration
    \end{itemize}
\end{enumerate}
\vspace{-1em}
\begin{figure}[t]
    \centering
    \begin{subfigure}{.3\textwidth}
        \centering
        \includegraphics[width=\textwidth]{images/Handshaking.pdf}
        \vspace{1em}
        \caption{PE-to-PE interface}
        \label{sfig:handshaking}
    \end{subfigure}
    \hspace{3em}
    \begin{subfigure}{.4\textwidth}
        \centering
        \includegraphics[width=.75\textwidth]{images/Predication.pdf}
        \vspace{-.5em}
        \caption{ALU with predication}
        \label{sfig:predication}
    \end{subfigure}
\end{figure}
\vspace{-1em}
\end{frame}

\begin{frame}{Turning Ideas and Plans into Reality}
\begin{centering}
    Open to collaboration - just reach out!

    \href{https://www.linkedin.com/in/hans-jakob-damsgaard/}{LinkedIn}
    \hspace{2em}
    \href{https://scholar.google.com/citations?user=xZrzxjQAAAAJ}{Scholar}
    \hspace{2em}
    \href{https://www.github.com/hansemandse}{GitHub}

\end{centering}
\end{frame}

\begin{frame}{Summary}
\begin{itemize}
    \item Chisel is powerful also for research
    \begin{itemize}
        \item Concise, flexible generators
        \item Single-language (family) development
        \item Design and implementation time
    \end{itemize}
    \vspace{1em}
    \item Define/decide on your personal design strategy
    \vspace{1em}
    \item Document your code; plenty of comments!
\end{itemize}
\end{frame}

\begin{frame}{Main Publications}
\begin{enumerate}
\item \textbf{Damsgaard}, Ometov, Nurmi. ``Approximation Opportunities in Edge Computing Hardware: A Systematic Literature Review.'' \textit{ACM CSUR 55.12 (2023)}: 1-49.
\item \textbf{Damsgaard}, Ometov, Nurmi. ``Verification of Approximate Hardware Designs with ChiselVerify.'' \textit{IEEE NorCAS}. 2023.
\item \textbf{Damsgaard}, Ometov, Nurmi. ``Generating CGRA Processing Element Hardware with CGRAgen.'' \textit{26\textsuperscript{th} Euromicro DSD.} IEEE, 2023.
\item \textbf{Damsgaard}, Ometov, Nurmi. ``Adaptive Approximate Computing with CGRAgen.'' \textit{IEEE Design \& Test 42.4 (2025)}: 41-47.
\end{enumerate}
\end{frame}

\begin{frame}{Funding Declaration}
\begin{centering}
    This lecture covered work conducted at Tampere University (TAU) within the APROPOS project. This project received funding from the European Union's Horizon 2020 Marie Sk\l{}odowska-Curie Innovative Training Networks under the Grant Agreement no. 956090.

\end{centering}
\vspace{1em}

\begin{centering}
    \url{https://apropos-itn.eu}

\end{centering}
\end{frame}



\section{Life as a PhD Student}
\begin{frame}{My Academic Path in Brief}
\begin{itemize}
    \item 2016-19 B.Sc. Electrical Engineering, DTU
    \vspace{1em}
    \item 2019-21 M.Sc. Computer Science and Engineering, DTU
    \begin{itemize}
        \item Exchange semester at UBC Vancouver
        \item Special course on instruction prefetching
        \item Special course on real-time NoC performance
    \end{itemize}
    \vspace{1em}
    \item 2021-24 Ph.D. Computing and Electrical Engineering, TAU
    \begin{itemize}
        \item One of 15 students in APROPOS
        \item Two-month research visit at IS-Wireless, Poland
        \item Six-month research visit at AMD Research, Germany
    \end{itemize}
\end{itemize}
\end{frame}

\note{Say something about my own experience with loneliness, the importance of a great supervisor, large degrees of self-control/motivation/passion for your topic, and the ability to limit oneself. Let the last point carry over into the next "section".}

\begin{frame}{Common Misconceptions}
\begin{itemize}
    \onslide<1->{
        \item The PhD journey is a lonesome ride
    }
    \vspace{1em}
    \onslide<2->{
        \item PhD students have no social life or time off
    }
    \vspace{1em}
    \onslide<3->{
        \item Your research must be \textit{revolutionizing}
    }
    \vspace{1em}
    \onslide<4->{
        \item A PhD degree limits your career options
    }
\end{itemize}
\vspace{2em}
\begin{centering}
    \onslide<5->{{\Large No!}\\(for the most part, at least)}

\end{centering}
\end{frame}

\begin{frame}{Common Misconceptions}
\begin{itemize}
    \item The PhD journey is a lonesome ride\\\onslide<2->{\quad$\Rightarrow$ Get yourself a great supervisor!}
    \item PhD students have no social life or time off\\\onslide<3->{\quad$\Rightarrow$ Some (few) do not; most do.}
    \item Your research must be \textit{revolutionizing}\\\onslide<4->{\quad$\Rightarrow$ It should be novel -- small steps are valuable too.}
    \item A PhD degree limits your career options\\\onslide<5->{\quad$\Rightarrow$ Depends on collective agreements, but generally not.}
\end{itemize}
\end{frame}

\note{A PhD is also a chance to travel, experience other cultures, meet interesting like-minded individuals (potential collaborators!), and geek out about techy stuff.}

\begin{frame}{Reality}
\framesubtitle{An Experience for Life}
\begin{itemize}
    \onslide<1->{
        \item First and foremost: a researcher education!
        \begin{itemize}
            \item Plan, structure, and carry-out scientific studies
            \item Report, publish, and present technical findings
        \end{itemize}
    }
    \vspace{1em}
    \onslide<2->{
        \item Study to become a field expert
        \begin{itemize}
            \item Contribute to advance field knowledge
            \item Build your personal resum\'e
        \end{itemize}
    }
\end{itemize}
\end{frame}



\section{Life after PhD Studies}
\begin{frame}{Meeting Industry}
\begin{itemize}
    \onslide<1->{
        \item Relatively minor experience thus far!
        \vspace{1em}
        \item 2024-25 SoC Verification Engineer, Nokia
        \begin{itemize}
            \item Large scale ASIC project; part of infrastructure IP team
            \item Worked in SystemVerilog, UVM, and C
        \end{itemize}
        \vspace{1em}
        \item 2025-{\color{white}25} Firmware Engineer, Teledyne RESON
        \begin{itemize}
            \item Smaller scale FPGA project; tightly integrated team
            \item Work in VHDL, Make, Tcl, and Java
        \end{itemize}
    }
    \vspace{1em}
    \onslide<2->{
        \item More well-defined tasks $\Rightarrow$ lower degree of freedom
        \item Some opposition to open-sourcing and publishing
    }
\end{itemize}
\end{frame}

\begin{frame}{What About Chisel?}
\begin{itemize}
    \onslide<1->{
        \item Major pros (from my perspective) include
        \begin{itemize}
            \item Rapid prototyping and (initial) testing
            \item Conciseness and readability
        \end{itemize}
    }
    \vspace{1em}
    \onslide<2->{
        \item Barriers met when promoting Chisel
        \begin{itemize}
            \item More languages (on top of RTL, TB, build automation etc.)
            \item Loss of control from compiler dependencies
            \item Lack of tool integration
            \item Long-lived traditions
        \end{itemize}
    }
\end{itemize}
\vspace{1em}
\onslide<3->{
    \begin{centering}
        ... but maybe they are (partly) right?

    \end{centering}
}
\end{frame}

\begin{frame}{Modern VHDL and SystemVerilog}
\begin{itemize}
    \onslide<1->{
        \item The traditional HDLs are alive and developing\footnote{No intention to undermine this course, merely observing our field.\\\quad\;\; Generator languages tend to use only legacy language features.}
    }
    \vspace{1em}
    \onslide<2->{
        \item VHDL 2019 (\href{https://ieeexplore.ieee.org/document/8938196}{IEEE 1076-2019})
        \begin{itemize}
            \item Improved preprocessor, interfaces, inline conditionals
            \item Width inference, range types, type generics
            \item Enhanced protected types and list/map types for simulation
        \end{itemize}
    }
    \vspace{1em}
    \onslide<3->{
        \item SystemVerilog 2023 (\href{https://ieeexplore.ieee.org/document/10458102}{IEEE 1800-2023})
        \begin{itemize}
            \item Improved preprocessor, enhanced coverpoints, map functions
            \item Note, UVM is a separate standard (\href{https://ieeexplore.ieee.org/document/9195920}{IEEE 1800.2-2020})
        \end{itemize}
    }
\end{itemize}
\vspace{1em}
\onslide<4->{
    \begin{centering}
        Beware of tool support!

    \end{centering}
}
\end{frame}

\begin{frame}{Verification Concerns}
\begin{itemize}
    \onslide<1->{
        \item Formal property-based verification
        \begin{itemize}
            \item \href{https://www.doulos.com/knowhow/psl/}{PSL} from VHDL 2008; \href{https://www.doulos.com/knowhow/systemverilog/systemverilog-tutorials/systemverilog-assertions-tutorial/}{SVA} from SystemVerilog 2005
            \item Simulator support remains so-and-so (in my experience)
            \item Model checking infeasible for large(r) designs
        \end{itemize}
    }
    \vspace{1em}
    \onslide<2->{
        \item UVM dominates VHDL alternatives for ASICs; more even for FPGAs\footnote{Check out the \href{https://uobdv.github.io/Design-Verification/WilsonResearchGroupFunctionalVerificationStudy/2020-WRGFV-Study/2020-WrG-FV-Study-Webinar-Oct13.pdf}{Wilson Verification Survey}.}
        \begin{itemize}
            \item Highly flexible and reusable for well-defined interfaces
            \item Fantastic tools for huge SoC-like projects
            \item UVM has a steep learning curve - what about \href{https://osvvm.org/about-os-vvm}{OSVVM}?
        \end{itemize}
    }
\end{itemize}
\vspace{1em}
\onslide<3->{
    \begin{centering}
        And again, beware of tool support!

    \end{centering}
}
\end{frame}



\section{Introduction to Today's Lab}
\begin{frame}{Lab Session: CSR Generation}
\begin{itemize}
    \onslide<1->{
        \item What are CSRs?
        \begin{itemize}
            \item \textit{Control and Status Registers}
            \item Memory-mapped configuration, control, and monitoring
            \item All IP cores have CSRs
        \end{itemize}
    }
    \vspace{1em}
    \onslide<2->{
        \item Why are CSRs problematic?
        \begin{itemize}
            \item In brief: \textbf{dependencies}!
            \item Stakeholders use different views of the CSRs
            \item CSR descriptions may be non-static during development
            \item Changes must propagate to everyone
        \end{itemize}
    }
    \vspace{1em}
    \onslide<3->{
        \item What is the solution?
        \begin{itemize}
            \item Maintain \textit{one} formal description of CSRs
            \item Generate all other views (RTL, TB, SW, and documentation)
            \item Use easily composable components
        \end{itemize}
    }
\end{itemize}
\end{frame}

\begin{frame}{Lab Session: Describing CSRs}
\begin{itemize}
    \onslide<1->{
        \item How \textit{should} one maintain CSR descriptions?
        \begin{itemize}
            \item Using established (open) standards
            \item Stored alongside IP sources
            \item Check out \href{https://www.accellera.org/images/downloads/standards/systemrdl/SystemRDL_2.0_Jan2018.pdf}{SystemRDL 2.0} and \href{https://www.accellera.org/images/downloads/standards/ip-xact/IPXACT-2022_user_guide.pdf}{IP-XACT}
        \end{itemize}
    }
    \vspace{1em}
    \onslide<2->{
        \item How \textit{are} CSR descriptions maintained?
        \begin{itemize}
            \item Wild West of company-internal tools
            \item Custom specification languages and formats\\(think: CSVs, Excel sheets, Word documents etc.)
            \item Syntax-check-free generator scripts\\(popular examples: Python, Perl, Shell)
        \end{itemize}
    }
\end{itemize}
\vspace{1em}
\onslide<3->{
    \begin{centering}
        Translating CSR descriptions is \textbf{tedious} and \textbf{error-prone}\\
        (and adds \textbf{no value} to existing products)

    \end{centering}
}
\end{frame}

\begin{frame}{Lab Session: Generating CSR Hardware}
\begin{itemize}
    \item You will generate a CSR adapter (register interface)
    \item Bridges between an interconnect and an IP core
    \item Assumes legacy sources CSR descriptions in Excel
    \item Relies on dynamic IO generation
\end{itemize}
\end{frame}

\section{Thank you for your attention!}

\end{document}
