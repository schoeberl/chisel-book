\documentclass[xcolor=pdflatex,dvipsnames,table]{beamer}
\usepackage{epsfig,graphicx}
\usepackage{palatino}
\usepackage{fancybox}
\usepackage{relsize}
\usepackage[procnames]{listings}
\usepackage{hyperref}
\usepackage{qtree} % needed?
\usepackage{booktabs}
\usepackage{dirtree}
\usepackage[normalem]{ulem}


% fatter TT font
\renewcommand*\ttdefault{txtt}
% another TT, suggested by Alex
% \usepackage{inconsolata}
% \usepackage[T1]{fontenc} % needed as well?


\newcommand{\scale}{0.7}

\newcommand{\todo}[1]{{\emph{TODO: #1}}}
\newcommand{\martin}[1]{{\color{blue} Martin: #1}}
\newcommand{\abcdef}[1]{{\color{red} Author2: #1}}

% uncomment following for final submission
%\renewcommand{\todo}[1]{}
%\renewcommand{\martin}[1]{}
%\renewcommand{\author2}[1]{}

\newcommand{\code}[1]{{\texttt{#1}}}

\hypersetup{
  linkcolor  = black,
%  citecolor  = blue,
  urlcolor   = blue,
  colorlinks = true,
}

\beamertemplatenavigationsymbolsempty
\setbeamertemplate{footline}[frame number]





\newif\ifbook
% shared in slides and book

\lstdefinelanguage{chisel}{
  morekeywords={abstract,case,catch,class,def,%
    do,else,extends,false,final,finally,%
    for,if,implicit,import,match,mixin,%
    new,null,object,override,package,%
    private,protected,requires,return,sealed,%
    super,this,throw,trait,true,try,%
    type,val,var,while,with,yield},
  otherkeywords={=>,<-,<\%,<:,>:,\#,@},
  sensitive=true,
  morecomment=[l]{//},
  morecomment=[n]{/*}{*/},
  morestring=[b]",
  morestring=[b]',
  morestring=[b]"""
}

\usepackage{color}
\definecolor{dkgreen}{rgb}{0,0.6,0}
\definecolor{gray}{rgb}{0.5,0.5,0.5}
\definecolor{mauve}{rgb}{0.58,0,0.82}

% Default settings for code listings
\ifbook
\lstset{%frame=lines,
  language=chisel,
  aboveskip=3mm,
  belowskip=3mm,
  showstringspaces=false,
  columns=fixed, % basewidth=\mybasewidth,
  basicstyle={\small\ttfamily},
  numbers=none,
  numberstyle=\footnotesize,
  % identifierstyle=\color{red},
  breaklines=true,
  breakatwhitespace=true,
  procnamekeys={def, val, var, class, trait, object, extends},
  % procnamestyle=\ttfamily,
  tabsize=2,
  float
}
\else
\lstset{%frame=lines,
  language=chisel,
  aboveskip=3mm,
  belowskip=3mm,
  showstringspaces=false,
  columns=fixed, % basewidth=\mybasewidth,
  basicstyle={\small\ttfamily},
  numbers=none,
  numberstyle=\footnotesize\color{gray},
  % identifierstyle=\color{red},
  keywordstyle=\color{blue},
  commentstyle=\color{dkgreen},
  stringstyle=\color{mauve},
  breaklines=true,
  breakatwhitespace=true,
  procnamekeys={def, val, var, class, trait, object, extends},
  procnamestyle=\ttfamily\color{red},
  tabsize=2,
  float
}
\fi

\lstnewenvironment{chisel}[1][]
{\lstset{language=chisel,#1}}
{}

\newcommand{\shortlist}[1]{{\lstinputlisting[nolol]{#1}}}

\newcommand{\longlist}[3]{{\lstinputlisting[float, caption={#2}, label={#3}, frame=tb, captionpos=b]{#1}}}

\newcommand{\verylonglist}[3]{{\lstinputlisting[caption={#2}, label={#3}, frame=tb, captionpos=b]{#1}}}


% TikZ for diagrams
\usepackage{tikz}
\usetikzlibrary{positioning, arrows.meta}

% Optional visual tweaks
\tikzset{
    >=Stealth,
    every node/.style={rounded corners=2pt}
}

\title{Generators}
\author{Martin Schoeberl}
\date{\today}
\institute{Technical University of Denmark\\
Embedded Systems Engineering}

\begin{document}

\begin{frame}
\titlepage
\end{frame}

\begin{frame}[fragile]{Overview}
\begin{itemize}
\item Functions with more outputs
\item Solution to the min/max lab
\item Type conversions
\item Parameters (simple and types)
\item More Scala for generators
\end{itemize}
\end{frame}

\begin{frame}[fragile]{Functions with Multiple Outputs}
\begin{itemize}
\item We use functions to generate hardware
\item The return value is the \emph{output} of that \emph{module}
\item A function usually has a single return value
\item Use a Scala tuple for more output \emph{ports}
\shortlist{../code/fun_comp.txt}
\end{itemize}
\end{frame}

\begin{frame}[fragile]{Functions with More Outputs}
\begin{itemize}
\item Access the two output wires with the \code{.\_n} syntax.
\shortlist{../code/fun_comp_use1.txt}
\item Or directly decompose the tuple
\shortlist{../code/fun_comp_use2.txt}
\item Functions can be declared as part of a \code{Module}
\item Better place them into a Scala object collecting utility functions
\item Functions can serve as ligthweight modules
\end{itemize}
\end{frame}

\begin{frame}[fragile]{Lab two Weeks Ago}
\begin{itemize}
\item Write a search for the minimum circuit (with \code{treeReduce()})
\item Add the generation of the index of the minimum value
\item This was a problem formulated by Microchip
\end{itemize}
\end{frame}

\begin{frame}[fragile]{Functional Generation}
\begin{itemize}
\item Anonymous functions, called \emph{function literal}
\begin{chisel}
  (param) => function body
\end{chisel}
\item A function for a minimum search
\shortlist{../code/fun_min.txt}
\item This was a very short exercise - let us extend this
\end{itemize}
\end{frame}


\begin{frame}[fragile]{Minimal Function with Index}
\begin{itemize}
\item Was the example for Tjark's heap sort
\shortlist{../code/fun_min2.txt}
\item We need an extra bundle to hold both values
\item A for loop is not so functional
\end{itemize}
\end{frame}

\begin{frame}[fragile]{We Can Use Tuples and zipWithIndex}
\begin{itemize}
\item \code{zipWithIndex} transforms the original sequence to a sequence of tuples with second element is the index
\item Use \code{map} to translate from Scala \code{Int} to Chisel \code{UInt}
\item \code{reduce} does the minimum function, actually called 2 times (we could optimize this)
\shortlist{../code/fun_min3.txt}
\item The result is a Scala \code{Vector} and not a Chisel \code{Vec}
\item No \code{reduceTree} available on a Scala \code{Vector}
\end{itemize}
\end{frame}

\begin{frame}[fragile]{Solution with a Vec}
\begin{itemize}
\item This results in a Chisel \code{Vec}
\shortlist{../code/fun_min4.txt}
\item \code{MixedVecInit} is like a \code{Bundle}, but indexable
\item We should add a \code{reduceTree} to the Scala sequence version (\code{TraversableOnce})
\end{itemize}
\end{frame}


\begin{frame}[fragile]{Type Conversion}
\begin{itemize}
\item Sometimes we would like to see a value in different types
\item All types represent a collection of bits
\item Example to package 4 bytes into a 32-bit \code{UInt}
\shortlist{../code/convert_vec2uint.txt}
\item And converting it back to a \code{Vec}
\shortlist{../code/convert_uint2vec.txt}
\end{itemize}
\end{frame}

\begin{frame}[fragile]{Type Conversion}
\begin{itemize}
\item Convert a \code{Bundle} to a \code{UInt}
\shortlist{../code/convert_bundle2uint.txt}
\item A \code{UInt} can be converted (back) to a bundle
\shortlist{../code/convert_uint2bundle.txt}
\item Initialize to 0 on a conversion
\shortlist{../code/convert_zero2bundle.txt}
\end{itemize}
\end{frame}


\begin{frame}[fragile]{Simple Parameters}
\begin{itemize}
\item Simplest way is bit width
\item You have seen this, also in Verilog or VHDL
\shortlist{../code/param_adder.txt}
\item A bit more interesting is using case classes for parameters
\shortlist{../code/case_class.txt}
\end{itemize}
\end{frame}



\begin{frame}[fragile]{Case Classes}
\begin{itemize}
\item Reading the immutable fields
\shortlist{../code/case_class_use.txt}
\item Adding checking code to case classes
\shortlist{../code/case_class_save.txt}
\end{itemize}
\end{frame}

\begin{frame}[fragile]{Module with Type Parameters}
\begin{itemize}
\item Assume a network-on-chip
\item Moves data between processing cores
\item We want to \emph{parameterize} that data type
\item Add a type parameter \code{T} to the Module constructor
\shortlist{../code/param_mod.txt}
\end{itemize}


\end{frame}
\begin{frame}[fragile]{Use that Router}
\begin{itemize}
\item Define data type we want to route with
\shortlist{../code/param_mod_type.txt}
\item Pass an instance of that bundle to the constructor of the router
\shortlist{../code/param_mod_use.txt}
\end{itemize}
\end{frame}




\begin{frame}[fragile]{Parameterized Bundles}
\begin{itemize}
\item We still have vectors of addresses and the payload
\item We want to parametrize a Bundle
\shortlist{../code/param_bundle_issue.txt}
\item \code{dt} is the type parameter, which we use for \code{cloneType}
\item However, it is a public field, in the way for using in a \code{Vec}
\end{itemize}
\end{frame}


\begin{frame}[fragile]{Parameterized Bundles}
\begin{itemize}
\item As a fix (workaround) make it private
\shortlist{../code/param_bundle.txt}
\item Define our router ports
\shortlist{../code/param_mod2.txt}
\end{itemize}
\end{frame}


\begin{frame}[fragile]{Using Parameterized Bundles}
\begin{itemize}
\item Instantiate that router with a \code{Port} that takes
a \code{Payload} as a parameter
\shortlist{../code/param_mod_use2.txt}
\end{itemize}
\end{frame}

\begin{frame}[fragile]{Scala Option}
\begin{itemize}
\item Scala's \code{Option[T]} is a wrapper around type T
\item Potential non-existence
\item Is either \code{Some(x)} or \code{None}
\begin{chisel}
val opt: Option[Int] = Some(123)
if (o.isDefined)
  println(o.get)
else
  println("None")
\end{chisel}
\end{itemize}
\end{frame}

\begin{frame}[fragile]{Optional Ports}
\begin{itemize}
\item IO ports may depend on configuration
\item In Scala, this is represented as an \code{Option}
\item Return a value wrapped in \code{Some} or represent the missing value
as a \code{None}
\item Could be used for debugging
\shortlist{../code/register_file_test.txt}
\end{itemize}
\end{frame}

\begin{frame}[fragile]{Example: Register File}
\begin{chisel}
class RegisterFile(debug: Boolean) extends Module {
  val io = IO(new Bundle {
    val rs1 = Input(UInt(5.W))
    val rs2 = Input(UInt(5.W))
...
    val rs2Val = Output(UInt(32.W))
    val dbgPort = if (debug)
      Some(Output(Vec(32, UInt(32.W)))) else None
  })
  val regfile = RegInit(VecInit(Seq.fill(32)(0.U(32.W))))
  io.rs1Val := regfile(io.rs1)
  io.rs2Val := regfile(io.rs2)
  when(io.wrEna) {
    regfile(io.rd) := io.wrData
  }
  if (debug) {
    io.dbgPort.get := regfile
  }}
\end{chisel}
\end{frame}

\begin{frame}[fragile]{Scala \code{tabulate}}
\begin{itemize}
\item More general than \code{fill}
\item Produce a new collection by calling an anonymous function
\item Index is the single argument
\item Can use the \_ wildcat
\begin{chisel}
Seq.fill(5)(0)
Seq.tabulate(5)(i => i * i)
Seq.tabulate(5)(_ + 1)
\end{chisel}
\end{itemize}
\end{frame}


\begin{frame}[fragile]{Scala's \code{apply()} Method}
\begin{itemize}
\item Create a new instance without \code{new}
\begin{chisel}
val p = Person("Jope Hacker")
\end{chisel}
\item is translated during compilation to
\begin{chisel}
val p = Person.apply("Jope Hacker")
\end{chisel}
\item \code{apply()} is also used as access function for arrays
\item No special syntax for arrays
\end{itemize}
\end{frame}

\begin{frame}[fragile]{Scala's \code{apply()} Method}
\begin{itemize}
\item A \emph{companion object} in Scala is an object
\begin{itemize}
\item Declared in the same file as a class,
\item and has the same name as the class
\end{itemize}
\item Add an \code{apply()} method to the companion object
\begin{chisel}
class Person {
    var name = ""
}

object Person {
    def apply(name: String): Person = {
        var p = new Person
        p.name = name
        p
    }
}
\end{chisel}
\end{itemize}
\end{frame}

\begin{frame}[fragile]{Factory Methods}
\begin{itemize}
\item Simpler component creation and use
\item Usage similar to built-in components, such as \code{Mux}
\end{itemize}
\begin{chisel}
val myAdder = Adder(x, y)
\end{chisel}
\begin{itemize}
\item A little bit more work on the component side
\item Define an \code{apply} method on the companion object that returns the component (output port)
\end{itemize}
\begin{chisel}
object Adder {
  def apply(a: UInt, b: UInt) = {
    val adder = Module(new Adder)
    adder.io.a := a
    adder.io.b := b
    adder.io.result
  }
}
\end{chisel}
\end{frame}

%\begin{frame}[fragile]{XXX}
%\begin{itemize}
%\item TODO: s4noc connection is part of the generator story
%\item
%\end{itemize}
%\end{frame}


\begin{frame}[fragile]{Summary}
\begin{itemize}
\item We use Scala to write hardware generators
\item Get started with your project
\item Next week: guest lecture by Emad on testing and CI
\end{itemize}
\end{frame}


\end{document}

\begin{frame}[fragile]{Title}
\begin{itemize}
\item abc
\end{itemize}
\end{frame}
