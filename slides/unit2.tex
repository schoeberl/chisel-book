\documentclass[xcolor=pdflatex,dvipsnames,table]{beamer}
\usepackage{epsfig,graphicx}
\usepackage{palatino}
\usepackage{fancybox}
\usepackage{relsize}
\usepackage[procnames]{listings}
\usepackage{hyperref}


% fatter TT font
\renewcommand*\ttdefault{txtt}
% another TT, suggested by Alex
% \usepackage{inconsolata}
% \usepackage[T1]{fontenc} % needed as well?

\usepackage[procnames]{listings}

% shared in slides and book

\lstdefinelanguage{chisel}{
  morekeywords={abstract,case,catch,class,def,%
    do,else,extends,false,final,finally,%
    for,if,implicit,import,match,mixin,%
    new,null,object,override,package,%
    private,protected,requires,return,sealed,%
    super,this,throw,trait,true,try,%
    type,val,var,while,with,yield},
  otherkeywords={=>,<-,<\%,<:,>:,\#,@},
  sensitive=true,
  morecomment=[l]{//},
  morecomment=[n]{/*}{*/},
  morestring=[b]",
  morestring=[b]',
  morestring=[b]"""
}

\usepackage{color}
\definecolor{dkgreen}{rgb}{0,0.6,0}
\definecolor{gray}{rgb}{0.5,0.5,0.5}
\definecolor{mauve}{rgb}{0.58,0,0.82}

% Default settings for code listings
\ifbook
\lstset{%frame=lines,
  language=chisel,
  aboveskip=3mm,
  belowskip=3mm,
  showstringspaces=false,
  columns=fixed, % basewidth=\mybasewidth,
  basicstyle={\small\ttfamily},
  numbers=none,
  numberstyle=\footnotesize,
  % identifierstyle=\color{red},
  breaklines=true,
  breakatwhitespace=true,
  procnamekeys={def, val, var, class, trait, object, extends},
  % procnamestyle=\ttfamily,
  tabsize=2,
  float
}
\else
\lstset{%frame=lines,
  language=chisel,
  aboveskip=3mm,
  belowskip=3mm,
  showstringspaces=false,
  columns=fixed, % basewidth=\mybasewidth,
  basicstyle={\small\ttfamily},
  numbers=none,
  numberstyle=\footnotesize\color{gray},
  % identifierstyle=\color{red},
  keywordstyle=\color{blue},
  commentstyle=\color{dkgreen},
  stringstyle=\color{mauve},
  breaklines=true,
  breakatwhitespace=true,
  procnamekeys={def, val, var, class, trait, object, extends},
  procnamestyle=\ttfamily\color{red},
  tabsize=2,
  float
}
\fi

\lstnewenvironment{chisel}[1][]
{\lstset{language=chisel,#1}}
{}

\newcommand{\shortlist}[1]{{\lstinputlisting[nolol]{#1}}}

\newcommand{\longlist}[3]{{\lstinputlisting[float, caption={#2}, label={#3}, frame=tb, captionpos=b]{#1}}}

\newcommand{\verylonglist}[3]{{\lstinputlisting[caption={#2}, label={#3}, frame=tb, captionpos=b]{#1}}}


\hypersetup{
  linkcolor  = black,
%  citecolor  = blue,
  urlcolor   = blue,
  colorlinks = true,
}

\newcommand{\code}[1]{{\texttt{#1}}}

\beamertemplatenavigationsymbolsempty
\setbeamertemplate{footline}[frame number]

\newcommand{\todo}[1]{{\emph{TODO: #1}}}
\newcommand{\martin}[1]{{\color{blue} Martin: #1}}
\newcommand{\abcdef}[1]{{\color{red} Author2: #1}}

% uncomment following for final submission
%\renewcommand{\todo}[1]{}
%\renewcommand{\martin}[1]{}
%\renewcommand{\author2}[1]{}


\title{Chisel Background and a Little Bit of Scala}
\author{Martin Schoeberl}
\date{\today}
\institute{DTU Compute}

\begin{document}

\begin{frame}
\titlepage
\end{frame}

\begin{frame}[fragile]{Chisel and Scala}
\begin{itemize}
\item Chisel is a library written in Scala
\item Chisel code is Scala code
\item When it is run is \emph{generates} hardware
\begin{itemize}
\item Verilog for synthesize
\item C++ code for simulation
\end{itemize}
\item xxx
\item xxx
\end{itemize}
\end{frame}


\begin{frame}[fragile]{Chisel Tutorial from UCB}
\begin{itemize}
\item Collection of small exercises
\item Only in simulation, no hardware required (+/-)
\item All examples in \emph{one} design
\item
\item
\item
\item
\end{itemize}
\end{frame}

\begin{frame}[fragile]{Chisel Tutorial}
\begin{itemize}
\item Get the tutorial
\end{itemize}
\begin{chisel}
git clone https://github.com/ucb-bar/chisel-tutorial.git
cd chisel-tutorial
\end{chisel}
\begin{itemize}
\item Test the installation with a Hello World
\end{itemize}
\begin{chisel}
cd hello
make
\end{chisel}
\begin{itemize}
\item May take some time
\end{itemize}
\end{frame}

\begin{frame}[fragile]{Very Minimal Hello World}
\begin{chisel}
class Hello extends Module {
  val io = new Bundle { 
    val out = UInt(OUTPUT, 8)
  }
  io.out := UInt(42)
}
\end{chisel}
\begin{itemize}
\item Produces a single constant
\end{itemize}
\end{frame}

\begin{frame}[fragile]{Testing the Minimal Hello World}
\begin{chisel}
class HelloTests(c: Hello) extends Tester(c) {
  step(1)
  expect(c.io.out, 42)
}
\end{chisel}
\begin{itemize}
\item Drive the simulation with \code{step(1)}, which is a single clock tick
\item Test output against expected value
\end{itemize}
\end{frame}

%%%%%%%%%%%%%%%%
\begin{frame}[fragile]{Title}
\begin{chisel}
code
\end{chisel}
\begin{itemize}
\item xxx
\item xxx
\item xxx
\end{itemize}
\end{frame}

\begin{frame}[fragile]{Title}
\begin{itemize}
\item xxx
\item xxx
\item xxx
\item xxx
\item xxx
\end{itemize}
\end{frame}
%%%%%%%%%%%%%%%%

\begin{frame}[fragile]{From here on Slides that shall be moved somewhere}
\end{frame}

\begin{frame}[fragile]{More Chisel Example Code}
\begin{itemize}
\item The time-predictable processor Patmos
\item An SRAM controller for the DE2-115 board
\item An SSRAM controller
\item An UART
\item A memory arbiter
\item Caches
\item ...
\item \url{https://github.com/t-crest/patmos}
\end{itemize}
\end{frame}


\begin{frame}[fragile]{More Chisel Documentation}
\begin{itemize}
\item Started a textbook on ``Digital Design with Chisel''
\item Considered a work-in-progress (V 0.01 ;-)
\item \url{https://github.com/schoeberl/chisel-book}
\item Will grow during the semester
\item Feedback is welcome
\end{itemize}
\end{frame}

\end{document}