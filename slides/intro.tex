\documentclass[xcolor=pdflatex,dvipsnames,table]{beamer}
\usepackage{epsfig,graphicx}
\usepackage{palatino}
\usepackage{fancybox}
\usepackage{relsize}
\usepackage[procnames]{listings}
\usepackage{hyperref}


% fatter TT font
\renewcommand*\ttdefault{txtt}
% another TT, suggested by Alex
% \usepackage{inconsolata}
% \usepackage[T1]{fontenc} % needed as well?

\usepackage[procnames]{listings}

% shared in slides and book

\lstdefinelanguage{chisel}{
  morekeywords={abstract,case,catch,class,def,%
    do,else,extends,false,final,finally,%
    for,if,implicit,import,match,mixin,%
    new,null,object,override,package,%
    private,protected,requires,return,sealed,%
    super,this,throw,trait,true,try,%
    type,val,var,while,with,yield},
  otherkeywords={=>,<-,<\%,<:,>:,\#,@},
  sensitive=true,
  morecomment=[l]{//},
  morecomment=[n]{/*}{*/},
  morestring=[b]",
  morestring=[b]',
  morestring=[b]"""
}

\usepackage{color}
\definecolor{dkgreen}{rgb}{0,0.6,0}
\definecolor{gray}{rgb}{0.5,0.5,0.5}
\definecolor{mauve}{rgb}{0.58,0,0.82}

% Default settings for code listings
\ifbook
\lstset{%frame=lines,
  language=chisel,
  aboveskip=3mm,
  belowskip=3mm,
  showstringspaces=false,
  columns=fixed, % basewidth=\mybasewidth,
  basicstyle={\small\ttfamily},
  numbers=none,
  numberstyle=\footnotesize,
  % identifierstyle=\color{red},
  breaklines=true,
  breakatwhitespace=true,
  procnamekeys={def, val, var, class, trait, object, extends},
  % procnamestyle=\ttfamily,
  tabsize=2,
  float
}
\else
\lstset{%frame=lines,
  language=chisel,
  aboveskip=3mm,
  belowskip=3mm,
  showstringspaces=false,
  columns=fixed, % basewidth=\mybasewidth,
  basicstyle={\small\ttfamily},
  numbers=none,
  numberstyle=\footnotesize\color{gray},
  % identifierstyle=\color{red},
  keywordstyle=\color{blue},
  commentstyle=\color{dkgreen},
  stringstyle=\color{mauve},
  breaklines=true,
  breakatwhitespace=true,
  procnamekeys={def, val, var, class, trait, object, extends},
  procnamestyle=\ttfamily\color{red},
  tabsize=2,
  float
}
\fi

\lstnewenvironment{chisel}[1][]
{\lstset{language=chisel,#1}}
{}

\newcommand{\shortlist}[1]{{\lstinputlisting[nolol]{#1}}}

\newcommand{\longlist}[3]{{\lstinputlisting[float, caption={#2}, label={#3}, frame=tb, captionpos=b]{#1}}}

\newcommand{\verylonglist}[3]{{\lstinputlisting[caption={#2}, label={#3}, frame=tb, captionpos=b]{#1}}}


\hypersetup{
  linkcolor  = black,
%  citecolor  = blue,
  urlcolor   = blue,
  colorlinks = true,
}


\newcommand{\todo}[1]{{\emph{TODO: #1}}}
\newcommand{\martin}[1]{{\color{blue} Martin: #1}}
\newcommand{\abcdef}[1]{{\color{red} Author2: #1}}

% uncomment following for final submission
%\renewcommand{\todo}[1]{}
%\renewcommand{\martin}[1]{}
%\renewcommand{\author2}[1]{}


\title{Digital Design in the 21st Century: Chisel}
\author{Martin Schoeberl}
\date{\today}
\institute{DTU Compute}

\begin{document}

\begin{frame}
\titlepage
\end{frame}

\begin{frame}{Goals for this Intro}
\begin{itemize}
\item Get an idea what Chisel is
\item Reconsider how to describe hardware
\item Some experience report from usage at DTU
\item Know where to get more information
\end{itemize}
\end{frame}

\begin{frame}{Talk abstract}

Date: Tu 19/04/2016, 11:00-12:00
Room: 170/324

Title: Hardware Design in the 21st Century: with the Object Oriented
and Functional Language Chisel

Chisel is a hardware construction language implemented as a
domain specific language in Scala. Therefore, the full power of
a modern programming language is available to describe hardware
and, more important, hardware generators. Chisel has been developed
at UC Berkeley and successfully used for several tape outs of RISC-V.
Here at DTU we used Chisel in the T-CREST project and in teaching
advanced computer architecture. Besides presenting small code
examples in Chisel I will report on experiences on using Chisel in
the t-CREST project and in teaching.

Martin Schoeberl
\end{frame}

\begin{frame}{yyy}
\begin{itemize}
\item xxx
\item xxx
\item xxx
\item xxx
\item xxx
\end{itemize}
\end{frame}

\begin{frame}[fragile]{Hello World in Chisel}
\begin{chisel}
class Hello extends Module {
  val io = new Bundle {
    val led = UInt(OUTPUT, 1)
  }
  val CNT_MAX = UInt(16000000 / 2 - 1);
  val r1 = Reg(init = UInt(0, 25))
  val blk = Reg(init = UInt(0, 1))

  r1 := r1 + UInt(1)
  when(r1 === CNT_MAX) {
    r1 := UInt(0)
    blk := ~blk
  }
  io.led := blk
}
\end{chisel}
\end{frame}



\end{document}