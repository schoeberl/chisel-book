\documentclass[xcolor=pdflatex,dvipsnames,table]{beamer}
\usepackage{epsfig,graphicx}
\usepackage{palatino}
\usepackage{fancybox}
\usepackage{relsize}
\usepackage[procnames]{listings}
\usepackage{hyperref}
\usepackage{qtree} % needed?
\usepackage{booktabs}
\usepackage{dirtree}
\usepackage[normalem]{ulem}


% fatter TT font
\renewcommand*\ttdefault{txtt}
% another TT, suggested by Alex
% \usepackage{inconsolata}
% \usepackage[T1]{fontenc} % needed as well?


\newcommand{\scale}{0.7}

\newcommand{\todo}[1]{{\emph{TODO: #1}}}
\newcommand{\martin}[1]{{\color{blue} Martin: #1}}
\newcommand{\abcdef}[1]{{\color{red} Author2: #1}}

% uncomment following for final submission
%\renewcommand{\todo}[1]{}
%\renewcommand{\martin}[1]{}
%\renewcommand{\author2}[1]{}

\newcommand{\code}[1]{{\texttt{#1}}}

\hypersetup{
  linkcolor  = black,
%  citecolor  = blue,
  urlcolor   = blue,
  colorlinks = true,
}

\beamertemplatenavigationsymbolsempty
\setbeamertemplate{footline}[frame number]





\newif\ifbook
% shared in slides and book

\lstdefinelanguage{chisel}{
  morekeywords={abstract,case,catch,class,def,%
    do,else,extends,false,final,finally,%
    for,if,implicit,import,match,mixin,%
    new,null,object,override,package,%
    private,protected,requires,return,sealed,%
    super,this,throw,trait,true,try,%
    type,val,var,while,with,yield},
  otherkeywords={=>,<-,<\%,<:,>:,\#,@},
  sensitive=true,
  morecomment=[l]{//},
  morecomment=[n]{/*}{*/},
  morestring=[b]",
  morestring=[b]',
  morestring=[b]"""
}

\usepackage{color}
\definecolor{dkgreen}{rgb}{0,0.6,0}
\definecolor{gray}{rgb}{0.5,0.5,0.5}
\definecolor{mauve}{rgb}{0.58,0,0.82}

% Default settings for code listings
\ifbook
\lstset{%frame=lines,
  language=chisel,
  aboveskip=3mm,
  belowskip=3mm,
  showstringspaces=false,
  columns=fixed, % basewidth=\mybasewidth,
  basicstyle={\small\ttfamily},
  numbers=none,
  numberstyle=\footnotesize,
  % identifierstyle=\color{red},
  breaklines=true,
  breakatwhitespace=true,
  procnamekeys={def, val, var, class, trait, object, extends},
  % procnamestyle=\ttfamily,
  tabsize=2,
  float
}
\else
\lstset{%frame=lines,
  language=chisel,
  aboveskip=3mm,
  belowskip=3mm,
  showstringspaces=false,
  columns=fixed, % basewidth=\mybasewidth,
  basicstyle={\small\ttfamily},
  numbers=none,
  numberstyle=\footnotesize\color{gray},
  % identifierstyle=\color{red},
  keywordstyle=\color{blue},
  commentstyle=\color{dkgreen},
  stringstyle=\color{mauve},
  breaklines=true,
  breakatwhitespace=true,
  procnamekeys={def, val, var, class, trait, object, extends},
  procnamestyle=\ttfamily\color{red},
  tabsize=2,
  float
}
\fi

\lstnewenvironment{chisel}[1][]
{\lstset{language=chisel,#1}}
{}

\newcommand{\shortlist}[1]{{\lstinputlisting[nolol]{#1}}}

\newcommand{\longlist}[3]{{\lstinputlisting[float, caption={#2}, label={#3}, frame=tb, captionpos=b]{#1}}}

\newcommand{\verylonglist}[3]{{\lstinputlisting[caption={#2}, label={#3}, frame=tb, captionpos=b]{#1}}}


\title{Finite-State Machines}
\author{Martin Schoeberl}
\date{\today}
\institute{Technical University of Denmark\\
Embedded Systems Engineering}

\begin{document}

\begin{frame}
\titlepage
\end{frame}


\begin{frame}[fragile]{TODO}
\begin{itemize}
\item Prepare answers for the midterm evaluation
\item State machine stuff (I should have PPT slides on this!)
\end{itemize}
\end{frame}

\begin{frame}[fragile]{Overview}
\begin{itemize}
\item Debugging with waveforms
\item Fun with counters
\item Finite-state machines
\item Collection with \code{Vec}
\end{itemize}
\end{frame}

\begin{frame}[fragile]{xxx}
\begin{itemize}
\item yyy
\end{itemize}
\end{frame}

\begin{frame}[fragile]{Testing with Chisel}
\begin{itemize}
\item Set input values with \code{poke}
\item Advance the simulation with \code{step}
\item Read the output values with \code{peek}
\item Compare the values with \code{expect}
\item Import following packages:
\shortlist{../code/test_import.txt}
\end{itemize}
\end{frame}

\begin{frame}[fragile]{Using \code{peek}, \code{poke}, and \code{expect}}
\begin{chisel}
// Set input values
poke(dut.io.a, 3)
poke(dut.io.b, 4)
// Execute one iteration
step(1)
// Print the result
val res = peek(dut.io.result)
println(res)

// Or compare against expected value
expect(dut.io.result, 7)
\end{chisel}
\end{frame}

\begin{frame}[fragile]{A Chisel Tester}
\begin{itemize}
\item Extends class \code{PeekPokeTester}
\item Has the device-under test (DUT) as parameter
\item Testing code can use all features of Scala
\end{itemize}
\begin{chisel}
class CounterTester(dut: Counter) extends PeekPokeTester(dut) {

  // Here comes the Chisel/Scala code
  // for the testing
}
\end{chisel}
\end{frame}

\begin{frame}[fragile]{Using ScalaTest for our Tester}
\begin{itemize}
\item Little verbose syntax
\item Copy example code to start with
\end{itemize}
\shortlist{../code/scalatest_simple.txt}
\end{frame}

\begin{frame}[fragile]{Generating Waveforms}
\begin{itemize}
\item Waveforms are timing diagrams
\item Good to see many parallel signals and registers
\item Additional parameters: \code{"--generate-vcd-output", "on"}
\item IO signals and registers are dumped
\item Generates a .vcd file
\item Viewing with GTKWave or ModelSim
\end{itemize}
\end{frame}

\begin{frame}[fragile]{A Self-Running Circuit}
\begin{itemize}
\item \code{SevenSegTest} is a self-running circuit
\item Needs no stimuli (\code{poke})
\item Just run for a few cycles
\item Tester for today's lab
\item This tester does NOT test the circuit
\begin{itemize}
\item You are not finished when this test does not show an error
\end{itemize}
\end{itemize}
\begin{chisel}
class SevenSegTest(dut: CountSevenSeg) extends PeekPokeTester(dut) {
  step(100)
}
\end{chisel}
\end{frame}

\begin{frame}[fragile]{Display Waveform with GTKWave}
\begin{itemize}
\item Run the tester: \code{sbt test}
\item Locate the .vcd file
\item Start GTKWave
\item Open the .vcd file with
\begin{itemize}
\item File -- Open New Tab
\end{itemize}
\item Select the circuit
\item Drag and drop the interesting signals
\end{itemize}
\end{frame}



\begin{frame}[fragile]{Call the Tester}
\begin{itemize}
\item Using here ScalaTest
\item Note \code{Driver.execute}
\item Note \code{Array("--generate-vcd-output", "on")}
\end{itemize}
\begin{chisel}
class SevenSegCountSpec extends
  FlatSpec with Matchers {
  "SevenSegTest " should "pass" in {
  chisel3.iotesters.Driver
  .execute(Array("--generate-vcd-output", "on"),
    () => new CountSevenSeg)
    { c => new SevenSegTest(c)} should be (true) }
}
\end{chisel}
\end{frame}




\begin{frame}[fragile]{Counters as Building Blocks}
\begin{itemize}
\item Counters are versatile tools
\item Count events
\item Generate timing ticks
\item Generate a one-shot timer
\end{itemize}
\end{frame}

\begin{frame}[fragile]{Counting Up and Down}
\begin{itemize}
\item Up:
\shortlist{../code/when_counter.txt}
\item Down:
\shortlist{../code/down_counter.txt}
\end{itemize}
\end{frame}

\begin{frame}[fragile]{Generating Timing with Counters}
\begin{itemize}
\item Generate a \code{tick} at a lower frequency
\item We used it in Lab 1 for the blinking LED
\item Use it for driving the display multiplexing at 1~kHz
\end{itemize}
\begin{figure}
  \includegraphics[scale=0.8]{../figures/tick_wave}
\end{figure}
\end{frame}


\begin{frame}[fragile]{The Tick Generation}
\shortlist{../code/sequ_tick_gen.txt}
\end{frame}

\begin{frame}[fragile]{Using the Tick}
\begin{itemize}
\item A counter running at a \emph{slower frequency}
\item By using the \code{tick} as an enable signal
\end{itemize}
\shortlist{../code/sequ_tick_counter.txt}
\end{frame}

\begin{frame}[fragile]{The \emph{Slow} Counter}
\begin{itemize}
\item Incremented every \code{tick}
\end{itemize}
\begin{figure}
  \includegraphics[scale=0.8]{../figures/tick_count_wave}
\end{figure}
\end{frame}

\begin{frame}[fragile]{A Timer}
\begin{itemize}
\item Like a kitchen timer
\item Start by loading a timeout value
\item Count down till 0
\item Assert \code{done} when finished
\end{itemize}
\end{frame}

\begin{frame}[fragile]{One-Shot Timer}
\begin{figure}
  \includegraphics[scale=\scale]{../figures/timer}
\end{figure}
\end{frame}

\begin{frame}[fragile]{One-Shot Timer}
\shortlist{../code/timer.txt}
\end{frame}

\begin{frame}[fragile]{A 4 Stage Shift Register}
\begin{figure}
  \includegraphics[scale=\scale]{../figures/shiftregister}
\end{figure}
\shortlist{../code/shift_register.txt}
\end{frame}


\begin{frame}[fragile]{A Shift Register with Parallel Output}
\begin{figure}
  \includegraphics[scale=\scale]{../figures/shiftreg-paraout}
\end{figure}
\shortlist{../code/shift_paraout.txt}
\end{frame}

\begin{frame}[fragile]{A Shift Register with Parallel Load}
\begin{figure}
  \includegraphics[scale=0.5]{../figures/shiftreg-paraload}
\end{figure}
\shortlist{../code/shift_paraload.txt}
\end{frame}


\begin{frame}[fragile]{A Simple Circuit}
\begin{itemize}
\item What does the following circuit?
\end{itemize}
\begin{figure}
  \includegraphics[scale=\scale]{../figures/fsm-rising}
\end{figure}
\end{frame}

\begin{frame}[fragile]{Finite-State Machine (FSM)}
\begin{itemize}
\item Has a register that contains the state
\item Has a function to computer the next state
\begin{itemize}
\item Depending on current state and input
\end{itemize}
\item Has an output depending on the state
\begin{itemize}
\item And maybe on the input as well
\end{itemize}
\item Every synchronous circuit can be considered a finite state machine
\item However, sometimes the state space is a little bit too large
\end{itemize}
\end{frame}

\begin{frame}[fragile]{Basic Finite-State Machine}
\begin{itemize}
\item A state register
\item Two combinational blocks
\end{itemize}
\begin{figure}
  \includegraphics[scale=\scale]{../figures/fsm}
\end{figure}
\end{frame}

\begin{frame}[fragile]{State Diagram}
\begin{figure}
  \includegraphics[scale=\scale]{../figures/state-diag-moore}
\end{figure}
\begin{itemize}
\item States and transitions depending on input values
\item Example is a simple alarm FSM
\item Nice visualization
\item Will not work for large FSMs
\end{itemize}
\end{frame}


\begin{frame}[fragile]{xxx}
\begin{itemize}
\item yyy
\end{itemize}
\end{frame}
\begin{frame}[fragile]{xxx}
\begin{itemize}
\item yyy
\end{itemize}
\end{frame}

\begin{frame}[fragile]{Scala List for Enumeration}
\begin{chisel}
  val empty :: full :: Nil = Enum(2)
\end{chisel}
\begin{itemize}
\item Can be used in wires and registers
\item Symbols for a state machine
\end{itemize}
\end{frame}

\begin{frame}[fragile]{Finite State Machine}
\begin{chisel}
  val empty :: full :: Nil = Enum(2)
  val stateReg = RegInit(empty)
  val dataReg = RegInit(0.U(size.W))

  when(stateReg === empty) {
    when(io.enq.write) {
      stateReg := full
      dataReg := io.enq.din
    }
  }.elsewhen(stateReg === full) {
    when(io.deq.read) {
      stateReg := empty
    }
  }
\end{chisel}
\begin{itemize}
\item A simple buffer for a bubble FIFO
\end{itemize}
\end{frame}


\begin{frame}[fragile]{xxx}
\begin{itemize}
\item yyy
\end{itemize}
\end{frame}

\begin{frame}[fragile]{A Collection of Signals with \code{Vec}}
\begin{itemize}
\item Chisel \code{Vec} is a collection of signals of the same type
\item The collection can be accessed by an index
\item Similar to an array in other languages
\item Wrap into a \code{Wire()} for combinational logic
\item Wrap into a \code{Reg()} for a collection of registers
\end{itemize}
\shortlist{../code/vec.txt}
\end{frame}

\begin{frame}[fragile]{Using a \code{Vec}}
\shortlist{../code/vec_access.txt}
\begin{itemize}
\item Reading from an \code{Vec} is a multplexer
\item We can put a \code{Vec} into a \code{Reg}
\end{itemize}
\shortlist{../code/reg_file.txt}
\noindent An element of that register file is accessed with an index and used as a normal register.

\shortlist{../code/reg_file_access.txt}
\end{frame}


\begin{frame}[fragile]{Mixing Vecs and Bundles}
\begin{itemize}
\item We can freely mix bundles and vectors
\item When creating a vector with a bundle
type, we need to pass a prototype for the vector fields. Using our
\code{Channel}, which we defined above, we can create a vector of channels with:
\end{itemize}
\shortlist{../code/vec_bundle.txt}
\begin{itemize}
\item A bundle may as well contain a vector
\end{itemize}
\shortlist{../code/bundle_vec.txt}
\end{frame}

\begin{frame}[fragile]{Today's Lab}
\begin{itemize}
\item Driving your 7-segment decoder
\item Use a counter to count from 0 to 15, driving your display
\item Use another counter to generate your timing
\begin{itemize}
\item We talked about this today
\end{itemize}
\item You clock on the board is 100 MHz
\item The given tester does only generate a waveform, no testing
\item Use a different count value for waveform debugging
\item Then synthesize it for the FPGA
\item Show a TA your working design
\item \href{https://github.com/schoeberl/chisel-lab/tree/master/lab6}{Lab 6}
\end{itemize}
\end{frame}

\begin{frame}[fragile]{Summary}
\begin{itemize}
\item xxx
\end{itemize}
\end{frame}


\begin{frame}[fragile]{Chisel VHDL Comparison}
\begin{columns}
\column{0.5\textwidth}
\begin{chisel}
class DecodeExecute extends Bundle {
  val rs1 = UInt(32.W)
  val rs2 = UInt(32.W)
  val immVal = UInt(32.W)
  val aluOp = new AluOp()
}
\end{chisel}
\column{0.5\textwidth}
\begin{verbatim}
VHDL code here
\end{verbatim}
\end{columns}
Also show latch and and using a button as clock
\end{frame}




\end{document}

%\begin{frame}[fragile]{xxx}
%\begin{itemize}
%\item yyy
%\end{itemize}
%\end{frame}
