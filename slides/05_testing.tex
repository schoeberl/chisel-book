\documentclass[xcolor=pdflatex,dvipsnames,table]{beamer}
\usepackage{epsfig,graphicx}
\usepackage{palatino}
\usepackage{fancybox}
\usepackage{relsize}
\usepackage[procnames]{listings}
\usepackage{hyperref}
\usepackage{qtree} % needed?
\usepackage{booktabs}
\usepackage{dirtree}
\usepackage[normalem]{ulem}


% fatter TT font
\renewcommand*\ttdefault{txtt}
% another TT, suggested by Alex
% \usepackage{inconsolata}
% \usepackage[T1]{fontenc} % needed as well?


\newcommand{\scale}{0.7}

\newcommand{\todo}[1]{{\emph{TODO: #1}}}
\newcommand{\martin}[1]{{\color{blue} Martin: #1}}
\newcommand{\abcdef}[1]{{\color{red} Author2: #1}}

% uncomment following for final submission
%\renewcommand{\todo}[1]{}
%\renewcommand{\martin}[1]{}
%\renewcommand{\author2}[1]{}

\newcommand{\code}[1]{{\texttt{#1}}}

\hypersetup{
  linkcolor  = black,
%  citecolor  = blue,
  urlcolor   = blue,
  colorlinks = true,
}

\beamertemplatenavigationsymbolsempty
\setbeamertemplate{footline}[frame number]





\newif\ifbook
% shared in slides and book

\lstdefinelanguage{chisel}{
  morekeywords={abstract,case,catch,class,def,%
    do,else,extends,false,final,finally,%
    for,if,implicit,import,match,mixin,%
    new,null,object,override,package,%
    private,protected,requires,return,sealed,%
    super,this,throw,trait,true,try,%
    type,val,var,while,with,yield},
  otherkeywords={=>,<-,<\%,<:,>:,\#,@},
  sensitive=true,
  morecomment=[l]{//},
  morecomment=[n]{/*}{*/},
  morestring=[b]",
  morestring=[b]',
  morestring=[b]"""
}

\usepackage{color}
\definecolor{dkgreen}{rgb}{0,0.6,0}
\definecolor{gray}{rgb}{0.5,0.5,0.5}
\definecolor{mauve}{rgb}{0.58,0,0.82}

% Default settings for code listings
\ifbook
\lstset{%frame=lines,
  language=chisel,
  aboveskip=3mm,
  belowskip=3mm,
  showstringspaces=false,
  columns=fixed, % basewidth=\mybasewidth,
  basicstyle={\small\ttfamily},
  numbers=none,
  numberstyle=\footnotesize,
  % identifierstyle=\color{red},
  breaklines=true,
  breakatwhitespace=true,
  procnamekeys={def, val, var, class, trait, object, extends},
  % procnamestyle=\ttfamily,
  tabsize=2,
  float
}
\else
\lstset{%frame=lines,
  language=chisel,
  aboveskip=3mm,
  belowskip=3mm,
  showstringspaces=false,
  columns=fixed, % basewidth=\mybasewidth,
  basicstyle={\small\ttfamily},
  numbers=none,
  numberstyle=\footnotesize\color{gray},
  % identifierstyle=\color{red},
  keywordstyle=\color{blue},
  commentstyle=\color{dkgreen},
  stringstyle=\color{mauve},
  breaklines=true,
  breakatwhitespace=true,
  procnamekeys={def, val, var, class, trait, object, extends},
  procnamestyle=\ttfamily\color{red},
  tabsize=2,
  float
}
\fi

\lstnewenvironment{chisel}[1][]
{\lstset{language=chisel,#1}}
{}

\newcommand{\shortlist}[1]{{\lstinputlisting[nolol]{#1}}}

\newcommand{\longlist}[3]{{\lstinputlisting[float, caption={#2}, label={#3}, frame=tb, captionpos=b]{#1}}}

\newcommand{\verylonglist}[3]{{\lstinputlisting[caption={#2}, label={#3}, frame=tb, captionpos=b]{#1}}}


\title{Testing and Verification}
\author{Martin Schoeberl}
\date{\today}
\institute{Technical University of Denmark\\
Embedded Systems Engineering}

\begin{document}

\begin{frame}
\titlepage
\end{frame}


\begin{frame}[fragile]{TODO}
\begin{itemize}
\item Midterm evaluation
\item \url{https://forms.gle/wtvDrA4peD4oLvt16}
\item Show lab exercise 7-segment decoder
\end{itemize}
\end{frame}


\begin{frame}[fragile]{Below Stuff Moved Over}
\begin{itemize}
\item From tutorial
\item Maybe used in the future
\end{itemize}
\end{frame}

\begin{frame}[fragile]{A Collection of Signals with \code{Vec}}
\begin{itemize}
\item Chisel \code{Vec} is a collection of signals of the same type
\item The collection can be accessed by an index
\item Similar to an array in other languages
\end{itemize}
\shortlist{../code/vec.txt}
\end{frame}

\begin{frame}[fragile]{Using a \code{Vec}}
\shortlist{../code/vec_access.txt}
\begin{itemize}
\item Reading from an \code{Vec} is a multplexer
\item We can put a \code{Vec} into a \code{Reg}
\end{itemize}
\shortlist{../code/reg_file.txt}
\noindent An element of that register file is accessed with an index and used as a normal register.

\shortlist{../code/reg_file_access.txt}
\end{frame}


\begin{frame}[fragile]{Mixing Vecs and Bundles}
\begin{itemize}
\item We can freely mix bundles and vectors
\item When creating a vector with a bundle
type, we need to pass a prototype for the vector fields. Using our
\code{Channel}, which we defined above, we can create a vector of channels with:
\end{itemize}
\shortlist{../code/vec_bundle.txt}
\begin{itemize}
\item A bundle may as well contain a vector
\end{itemize}
\shortlist{../code/bundle_vec.txt}
\end{frame}

\begin{frame}[fragile]{Functional Abstraction}
\begin{chisel}
  def addSub(add: Bool, a: UInt, b: UInt) =
    Mux(add, a+b, a-b)

  val res = addSub(cond, a, b)
  
  def rising(d: Bool) = d && !RegNext(d)
\end{chisel}
\begin{itemize}
\item Functions for repeated pieces of logic
\item May contain state
\item Functions may return \emph{hardware}
\end{itemize}
\end{frame}

\begin{frame}[fragile]{Component Generation}
\begin{chisel}
val cores = new Array[Module](32)

for (j <- 0 until 32)
  cores(j) = Module(new CPU())
\end{chisel}
\begin{itemize}
\item Use Scala array to collect components
\item Generation with a Scala loop
\end{itemize}
\end{frame}

\begin{frame}[fragile]{Conditional Component Generation}
\begin{chisel}
val icache =
  if (TYPE == METHOD)
    Module(new MCache())
  else if (TYPE == LINE)
    Module(new ICache())
  else
    ChiselError.error("Unsupported Type")
\end{chisel}
\begin{itemize}
\item Use Scala if/else for conditional component types
\item Code example from Patmos
\item We parse an XML file for the configuration
\end{itemize}
\end{frame}

\begin{frame}[fragile]{Logic Generation}
\begin{itemize}
\item Read a file into a table
\begin{itemize}
\item E.g., to read in ROM content for a processor
\end{itemize}
\item Generate a truth table algorithmically
\begin{itemize}
\item E.g., generate binary to BCD translation
\end{itemize}
\item Use the full power of Scala
\end{itemize}
\begin{chisel}
val byteArray = Files.readAllBytes(Paths.get(fileName))
val arr = new Array[Bits](byteArray.length)
for (i <- 0 until byteArray.length) {
  arr(i) = Bits(byteArray(i), 8)
}
val rom = Vec[Bits](arr)
\end{chisel}
\end{frame}
%%%%%%%%%%%%%%%%%
%\begin{frame}[fragile]{zzz}
%\begin{chisel}
%code
%\end{chisel}
%\begin{itemize}
%\item xxx
%\item xxx
%\item xxx
%\end{itemize}
%\end{frame}
%
%\begin{frame}[fragile]{yyy}
%\begin{itemize}
%\item xxx
%\item xxx
%\item xxx
%\item xxx
%\item xxx
%\end{itemize}
%\end{frame}
%%%%%%%%%%%%%%%%%

\begin{frame}[fragile]{Scala Values and Variables}
\begin{chisel}
// A value is a constant
val i = 0
// No new assignment; this will not compile
i = 3

// A variable can change the value
var v = "Hello"
v = "Hello World"

// Type usually inferred, but can be declared
var s: String = "abc"
\end{chisel}
\end{frame}

\begin{frame}[fragile]{Simple Loops}
\begin{chisel}
// Loops from 0 to 9
// Automatically creates loop value i
for (i <- 0 until 10) {
  println(i)
}
\end{chisel}
\end{frame}

\begin{frame}[fragile]{Conditions}
\begin{chisel}
for (i <- 0 until 10) {
  if (i%2 == 0) {
    println(i + " is even")
  } else {
    println(i + " is odd")
  }
}
\end{chisel}
\end{frame}

\begin{frame}[fragile]{Scala Arrays and Lists}
\begin{chisel}
// An integer array with 10 elements
val numbers = new Array[Integer](10)
for (i <- 0 until numbers.length) {
  numbers(i) = i*10
}
println(numbers(9))


// List of integers
val list = List(1, 2, 3)
println(list)
// Different form of list construction
val listenum = 'a' :: 'b' :: 'c' :: Nil
println(listenum)
\end{chisel}
\end{frame}


\begin{frame}[fragile]{Scala Classes}
\begin{chisel}
// A simple class
class Example {
  // A field, initialized in the constructor
  var n = 0
  
  // A setter method
  def set(v: Integer) = {
    n = v
  }
  
  // Another method
  def print() = {
    println(n)
  }
}
\end{chisel}
\end{frame}

\begin{frame}[fragile]{Scala (Singleton) Object}
\begin{chisel}
object Example {}
\end{chisel}
\begin{itemize}
\item For \emph{static} fields and methods
\begin{itemize}
\item Scala has no static fields or methods like Java
\end{itemize}
\item Needed for \code{main}
\item Useful for helper functions
\end{itemize}
\end{frame}

\begin{frame}[fragile]{Singleton Object for the \code{main}}
\begin{chisel}
// A singleton object
object Example {
  
  // The start of a Scala program
  def main(args: Array[String]): Unit = {
    
    val e = new Example()
    e.print()
    e.set(42)
    e.print()
  }
}
\end{chisel}
\begin{itemize}
\item Compile and run it with sbt (or within Eclipse/IntelliJ):
\end{itemize}
\begin{chisel}
sbt "runMain Example"
\end{chisel}
\end{frame}

\begin{frame}[fragile]{Functions}
\begin{itemize}
\item Circuits can be encapsulated in functions
\item Each \emph{function call} generates hardware
\item Simple functions can be a single line
\end{itemize}
\begin{chisel}
  def adder(v1: UInt, v2: UInt) = v1 + v2
  
  val add1 = adder(a, b)
  val add2 = adder(c, d)
\end{chisel}
\end{frame}

\begin{frame}[fragile]{More Function Examples}
\begin{itemize}
\item Functions can also contain registers
\end{itemize}
\begin{chisel}
  def addSub(add: Bool, a: UInt, b: UInt) =
    Mux(add, a + b, a - b)

  val res = addSub(cond, a, b)

  def rising(d: Bool) = d && !RegNext(d)

  val edge = rising(cond)
\end{chisel}
\end{frame}

\begin{frame}[fragile]{The Counter as a Function}
\begin{itemize}
\item Longer functions in curly brackets
\item Last value is the return value
\end{itemize}
\begin{chisel}
def counter(n: UInt) = {
  
  val cntReg = RegInit(0.U(8.W))
  
  cntReg := cntReg + 1.U
  when(cntReg === n) {
    cntReg := 0.U
  }
  cntReg
}

val counter100 = counter(100.U)
\end{chisel}
\end{frame}


\begin{frame}[fragile]{Functions}
\begin{itemize}
\item Example from Patmos execute stage
\end{itemize}
\begin{chisel}
def alu(func: Bits, op1: UInt, op2: UInt): Bits = {
  val result = UInt(width = DATA_WIDTH)
  // some more lines...
  switch(func) {
    is(FUNC_ADD) { result := sum }
    is(FUNC_SUB) { result := op1 - op2 }
    is(FUNC_XOR) { result := (op1 ^ op2).toUInt }
    // some more lines
  }
  result
}
\end{chisel}
\end{frame}

\begin{frame}[fragile]{Scala List for Enumeration}
\begin{chisel}
  val empty :: full :: Nil = Enum(2)
\end{chisel}
\begin{itemize}
\item Can be used in wires and registers
\item Symbols for a state machine
\end{itemize}
\end{frame}

\begin{frame}[fragile]{Finite State Machine}
\begin{chisel}
  val empty :: full :: Nil = Enum(2)
  val stateReg = RegInit(empty)
  val dataReg = RegInit(0.U(size.W))

  when(stateReg === empty) {
    when(io.enq.write) {
      stateReg := full
      dataReg := io.enq.din
    }
  }.elsewhen(stateReg === full) {
    when(io.deq.read) {
      stateReg := empty
    }
  }
\end{chisel}
\begin{itemize}
\item A simple buffer for a bubble FIFO
\end{itemize}
\end{frame}

\begin{frame}[fragile]{Parameterization}
\begin{chisel}
class ParamChannel(n: Int) extends Bundle {
  val data = Input(UInt(n.W))
  val ready = Output(Bool())
  val valid = Input(Bool())
}

val ch32 = new ParamChannel(32)
\end{chisel}
\begin{itemize}
\item Bundles and modules can be parametrized
\item Pass a parameter in the constructor
\end{itemize}

\end{frame}
\begin{frame}[fragile]{A Module with a Parameter}
\begin{chisel}
class ParamAdder(n: Int) extends Module {
  val io = IO(new Bundle {
    val a = Input(UInt(n.W))
    val b = Input(UInt(n.W))
    val result = Output(UInt(n.W))
  })

  val addVal = io.a + io.b
  io.result := addVal
}

val add8 = Module(new ParamAdder(8))
\end{chisel}
\begin{itemize}
\item Parameter can also be a Chisel type
\item Can also be a generic type:
\item \code{class Mod[T <: Bits](param: T) extends...}
\end{itemize}
\end{frame}

\begin{frame}[fragile]{Scala \code{for} Loop for Circuit Generation}
\begin{chisel}
val shiftReg = RegInit(0.U(8.W))

shiftReg(0) := inVal

for (i <- 1 until 8) {
  shiftReg(i) := shiftReg(i-1)
}
\end{chisel}
\begin{itemize}
\item \code{for} is Scala
\item This loop generates several connections
\item The connections are parallel hardware
\end{itemize}
\end{frame}

\begin{frame}[fragile]{Conditional Circuit Generation}
\begin{chisel}
class Base extends Module { val io = new Bundle() }
class VariantA extends Base { }
class VariantB extends Base { }

val m = if (useA) Module(new VariantA())
        else Module(new VariantB())
\end{chisel}
\begin{itemize}
\item \code{if} and \code{else} is Scala
\item \code{if} is an expression that returns a value
\begin{itemize}
\item Like ``\code{cond ? a : b;}'' in C and Java
\end{itemize}
\item This is not a hardware multiplexer
\item Decides which module to generate
\item Could even read an XML file for the configuration
\end{itemize}
\end{frame}

%\begin{frame}[fragile]{Chisel has a Multiplexer}
\begin{figure}
  \includegraphics[scale=\scale]{../figures/mux}
\end{figure}
\shortlist{../code/mux.txt}
\begin{itemize}
\item So what?
\item Wait... What type is \code{a} and \code{b}?
\begin{itemize}
\item Can be any Chisel type!
\end{itemize}
\end{itemize}
\end{frame}

\begin{frame}[fragile]{Chisel has a Generic Multiplexer}
\begin{figure}
  \includegraphics[scale=\scale]{../figures/mux}
\end{figure}
\shortlist{../code/mux.txt}
\begin{itemize}
\item SW people may not be impressed
\item They have generics since Java 1.5 in 2004
\begin{itemize}
\item \code{List<Flowers> != List<Cars>}
\end{itemize}
\end{itemize}
\end{frame}


\begin{frame}[fragile]{Generics in Hardware Construction}
\begin{itemize}
\item Chisel supports generic classes with type parameters
\item Write hardware generators independent of concrete type
\item This is a multiplexer \emph{generator}
\end{itemize}
\shortlist{../code/param_func.txt}
\end{frame}

\begin{frame}[fragile]{Put Generics Into Use}
\begin{itemize}
\item Let us implement a generic FIFO
\item Use the generic ready/valid interface from Chisel
\end{itemize}
\shortlist{../code/fifo_decoupled.txt}
\end{frame}

\begin{frame}[fragile]{Define the FIFO Interface}
\shortlist{../code/fifo_io.txt}
\begin{itemize}
\item We need enqueueing and dequeueing ports
\item Note the \code{Flipped}
\begin{itemize}
\item It switches the direction of ports
\item No more double definitions of an interface
\end{itemize}
\end{itemize}
\end{frame}

\begin{frame}[fragile]{But What FIFO Implementation?}
\begin{itemize}
\item Bubble FIFO (good for low data rate)
\item Double buffer FIFO (fast restart)
\item FIFO with memory and pointers (for larger buffers)
\begin{itemize}
\item Using flip-flops
\item Using on-chip memory
\end{itemize}
\item And some more...
\end{itemize}
\begin{itemize}
\item This calls for object-oriented \sout{programming} \emph{hardware construction}
\end{itemize}
\end{frame}

\begin{frame}[fragile]{Abstract Base Class and Concrete Extension}
\shortlist{../code/fifo_abstract.txt}
\begin{itemize}
\item May contain common code
\item Extend by concrete classes
\end{itemize}
\begin{chisel}
class BubbleFifo[T <: Data](gen: T, depth: Int) extends Fifo(gen: T, depth: Int) {
\end{chisel}
\end{frame}



\begin{frame}[fragile]{Select a Concrete FIFO Implementation}
\begin{itemize}
\item Decide at hardware generation
\item Can use all Scala/Java power for the decision
\begin{itemize}
\item Connect to a web service, get \sout{Google} Alphabet stock price, and decide on which to use ;-)
\item For sure a silly idea, but you see what is possible...
\item Developers may find clever use of the Scala/Java power
\item We could present a GUI to the user to select from
\end{itemize}
\item We use XML files parsed at hardware generation time
\item End of TCL, Python,... generated hardware
\end{itemize}
\end{frame}

\begin{frame}[fragile]{Binary to BCD Conversion for VHDL}
\begin{figure}
    \centering
    \includegraphics[scale=0.6]{JavaBCD}
\end{figure}
\end{frame}

\begin{frame}[fragile]{Java Program}
\begin{itemize}
\item Generates a VHDL table
\item The core code is:
\end{itemize}
\begin{chisel}
for (int i = 0; i < Math.pow(2, ADDRBITS); ++i) {
    int val = ((i/10)<<4) + i%10;
    // write out VHDL code for each line
\end{chisel}
\begin{itemize}
\item With all boilerplate 118 LoC
\end{itemize}
\end{frame}

\begin{frame}[fragile]{Chisel Version of Binary to BCD Conversion}
%\begin{chisel}
%  val array = new Array[Int](256)
%  for (i <- 0 to 99) {
%    array(i) = ((i/10)<<4) + i%10
%  }
%  val table = VecInit(array.map(_.U(8.W)))
%\end{chisel}
\begin{chisel}
  val table = Wire(Vec(100, UInt(8.W)))
  for (i <- 0 until 100) {
    table(i) := (((i/10)<<4) + i%10).U
  }
  val bcd = table(bin)
\end{chisel}
\begin{itemize}
\item Directly generates the hardware table as a \code{Vec}
\item At hardware construction time
\item In the same language
\end{itemize}
\end{frame}

\begin{frame}[fragile]{Use Functional Programming for Generators}
\shortlist{../code/fun_first.txt}
\shortlist{../code/fun_func_lit.txt}
\shortlist{../code/fun_reduce_tree.txt}
\begin{itemize}
\item This is a simple example
\item What about an arbiter tree with fair arbitration?
\end{itemize}
\end{frame}

%\begin{frame}[fragile]{XXX}
%\begin{itemize}
%\item TODO: s4noc connection is part of the generator story
%\item
%\item
%\end{itemize}
%\end{frame}
%
%\begin{frame}[fragile]{XXX}
%\begin{itemize}
%\item
%\item
%\item
%\end{itemize}
%\end{frame}



\begin{frame}[fragile]{Generation Slides (inlude) are missing here}
\begin{itemize}
\item 
\end{itemize}
\end{frame}

\begin{frame}[fragile]{Combinational (Truth) Table Generation}
\begin{chisel}
val arr = new Array[Bits](length)
for (i <- 0 until length) {
  arr(i) = ...
}
val rom = Vec[Bits](arr)
\end{chisel}
\begin{itemize}
\item Generate a table in a Scala array
\item Use that array as input for a Chisel \code{Vec}
\item Generates a logic table at hardware construction time
\end{itemize}
\end{frame}

\begin{frame}[fragile]{BCD Generation Example}
\begin{itemize}
\item Explain BCD with examples
\item Show code
\end{itemize}
\end{frame}

\begin{frame}[fragile]{Ideas for Runtime Table Generation}
\begin{itemize}
\item Assembler in Scala/Java generates the boot ROM
\item Table with a \code{sin} function
\item Binary to BCD conversion
\item Schedule table for a TDM based network-on-chip
\item 
\item More ideas?
\end{itemize}
\end{frame}

\begin{frame}[fragile]{Memory}
\begin{chisel}
val mem = Mem(Bits(width = 8), size)

// write
when(wrEna) {
  mem(wrAddr) := wrData
}

// read
val rdAddrReg = Reg(next = rdAddr)
rdData := mem(rdAddrReg)
\end{chisel}
\begin{itemize}
\item Write is synchronous
\item Read can be asynchronous or synchronous
\item But there are no asynchronous memories in an FPGA
\end{itemize}
\end{frame}

\begin{frame}[fragile]{Factory Methods}
\begin{itemize}
\item Simpler component creation and use
\item Usage similar to built in components, such as \code{Mux}
\end{itemize}
\begin{chisel}
val myAdder = Adder(x, y)
\end{chisel}
\begin{itemize}
\item A little bit more work on component side
\item Define an \code{apply} method on the companion object that returns the component
\end{itemize}
\begin{chisel}
object Adder {
  def apply(a: UInt, b: UInt) = {
    val adder = Module(new Adder)
    adder.io.a := a
    adder.io.b := b
    adder.io.result
  }
}
\end{chisel}
\end{frame}

\begin{frame}[fragile]{Summary}
\begin{itemize}
\item Chisel is a small language
\item Embedding it in Scala gives the power
\item We can write circuit generators
\item We can to co-simulation
\item We just scratched the surface
\end{itemize}
\end{frame}




\end{document}

