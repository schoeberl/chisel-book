\documentclass[xcolor=pdflatex,dvipsnames,table]{beamer}
\usepackage{epsfig,graphicx}
\usepackage{palatino}
\usepackage{fancybox}
\usepackage{relsize}
\usepackage[procnames]{listings}
\usepackage{hyperref}
\usepackage{qtree} % needed?
\usepackage{booktabs}
\usepackage{dirtree}
\usepackage[normalem]{ulem}


% fatter TT font
\renewcommand*\ttdefault{txtt}
% another TT, suggested by Alex
% \usepackage{inconsolata}
% \usepackage[T1]{fontenc} % needed as well?


\newcommand{\scale}{0.7}

\newcommand{\todo}[1]{{\emph{TODO: #1}}}
\newcommand{\martin}[1]{{\color{blue} Martin: #1}}
\newcommand{\abcdef}[1]{{\color{red} Author2: #1}}

% uncomment following for final submission
%\renewcommand{\todo}[1]{}
%\renewcommand{\martin}[1]{}
%\renewcommand{\author2}[1]{}

\newcommand{\code}[1]{{\texttt{#1}}}

\hypersetup{
  linkcolor  = black,
%  citecolor  = blue,
  urlcolor   = blue,
  colorlinks = true,
}

\beamertemplatenavigationsymbolsempty
\setbeamertemplate{footline}[frame number]





\newif\ifbook
% shared in slides and book

\lstdefinelanguage{chisel}{
  morekeywords={abstract,case,catch,class,def,%
    do,else,extends,false,final,finally,%
    for,if,implicit,import,match,mixin,%
    new,null,object,override,package,%
    private,protected,requires,return,sealed,%
    super,this,throw,trait,true,try,%
    type,val,var,while,with,yield},
  otherkeywords={=>,<-,<\%,<:,>:,\#,@},
  sensitive=true,
  morecomment=[l]{//},
  morecomment=[n]{/*}{*/},
  morestring=[b]",
  morestring=[b]',
  morestring=[b]"""
}

\usepackage{color}
\definecolor{dkgreen}{rgb}{0,0.6,0}
\definecolor{gray}{rgb}{0.5,0.5,0.5}
\definecolor{mauve}{rgb}{0.58,0,0.82}

% Default settings for code listings
\ifbook
\lstset{%frame=lines,
  language=chisel,
  aboveskip=3mm,
  belowskip=3mm,
  showstringspaces=false,
  columns=fixed, % basewidth=\mybasewidth,
  basicstyle={\small\ttfamily},
  numbers=none,
  numberstyle=\footnotesize,
  % identifierstyle=\color{red},
  breaklines=true,
  breakatwhitespace=true,
  procnamekeys={def, val, var, class, trait, object, extends},
  % procnamestyle=\ttfamily,
  tabsize=2,
  float
}
\else
\lstset{%frame=lines,
  language=chisel,
  aboveskip=3mm,
  belowskip=3mm,
  showstringspaces=false,
  columns=fixed, % basewidth=\mybasewidth,
  basicstyle={\small\ttfamily},
  numbers=none,
  numberstyle=\footnotesize\color{gray},
  % identifierstyle=\color{red},
  keywordstyle=\color{blue},
  commentstyle=\color{dkgreen},
  stringstyle=\color{mauve},
  breaklines=true,
  breakatwhitespace=true,
  procnamekeys={def, val, var, class, trait, object, extends},
  procnamestyle=\ttfamily\color{red},
  tabsize=2,
  float
}
\fi

\lstnewenvironment{chisel}[1][]
{\lstset{language=chisel,#1}}
{}

\newcommand{\shortlist}[1]{{\lstinputlisting[nolol]{#1}}}

\newcommand{\longlist}[3]{{\lstinputlisting[float, caption={#2}, label={#3}, frame=tb, captionpos=b]{#1}}}

\newcommand{\verylonglist}[3]{{\lstinputlisting[caption={#2}, label={#3}, frame=tb, captionpos=b]{#1}}}


\title{Finite State Machine with Datapath}
\author{Martin Schoeberl}
\date{\today}
\institute{Technical University of Denmark\\
Embedded Systems Engineering}

\begin{document}

\begin{frame}
\titlepage
\end{frame}


\begin{frame}[fragile]{TODO}
\begin{itemize}
\item Explain the paper and pencile (at the end)
\item Repeat the testing including showing it on the screen
\item Try it out with some
\item add cosimulation to the verification story (show Lipsi), maybe later
\item logic gen., maybe bin bcd directly into vec
\end{itemize}
\end{frame}

\begin{frame}[fragile]{Overview}
\begin{itemize}
\item Remote teaching and learning
\item Repeat
\item FSM with Datapaht
\end{itemize}
\end{frame}

\begin{frame}[fragile]{Remote Learning}
\begin{itemize}
\item First: this is all new for the most of us
\begin{itemize}
\item We need to be patient with each other
\end{itemize}
\item We should use Slack for quicker communication
\item Zoom for lecturing
\begin{itemize}
\item \url{https://dtudk.zoom.us/j/7712109832}
\end{itemize}
\item Please use a headset
\item Please mute your mic when not talking
\item Some nice features
\begin{itemize}
\item You can raise your hand
\item You can ask questions with mic or on chat
\item Everyone can share their screen or an individual window
\end{itemize}
\item The Dally book is now available online for free!
\end{itemize}
\end{frame}

\begin{frame}[fragile]{Lab/Exercise Organization}
\begin{itemize}
\item Everyone at DTU can use the professional version of Zoom
\begin{itemize}
\item \url{http://dtudk.zoom.us/signin}
\end{itemize}
\item You can also use Zoom for your group work
\item Zoom will also be used for the supervised lab
\begin{itemize}
\item Start a Zoom meeting with your group
\item Schedule a TA for help with Slack, posting your Zoom meeting link
\item You can also schedule a Zoom meeting with me (also at other times)
\end{itemize}
\item This is a chance to learn how to collaborate remotely
\begin{itemize}
\item This will be part of your future work as engineer anyway
\end{itemize}
\item This experiment might change how we teach in the future
\end{itemize}
\end{frame}

\begin{frame}[fragile]{Lab Work}
\begin{itemize}
\item We will stick to the plan of a working Vending Machine
\begin{itemize}
\item At the end it shall run in your FPGA board
\item I am a big fan of running stuff in real hardware
\end{itemize}
\item Demo your work to a TA via the camera
\item I know many groups have only one physical FPGA board
\item A lot can be done in simulation
\item I plan to develop a simulation of the Basys3 board
\item I assume you have found a solution for file sharing
\begin{itemize}
\item GitHub is a popular one for source code
\item Can also be used if you plan to write your report in LaTeX
\end{itemize}
\end{itemize}
\end{frame}

\begin{frame}[fragile]{Questions?}
\begin{itemize}
\item On lectures
\item On the group/lab work
\end{itemize}
\end{frame}

\begin{frame}[fragile]{FSM with Datapath}
\begin{itemize}
\item A type of computing machine
\item Consists of a finite-state machine (FSM) and a datapath
\item The FSM is the master (the controller) of the datapath
\item The datapath has computing elements
\begin{itemize}
\item E.g., adder, incrementer, constants, multiplexers, ...
\end{itemize}
\item The datapath has storage elements (registers)
\begin{itemize}
\item E.g., sum of money payed, count of something, ...
\end{itemize}
\end{itemize}
\end{frame}

\begin{frame}[fragile]{FSM-Datapath Interaction}
\begin{itemize}
\item The FSM controls the datapath
\begin{itemize}
\item For example, add 2 to the sum
\end{itemize}
\item By controlling multiplexers
\begin{itemize}
\item For example, select how much to add
\item Not adding means selecting 0 to add
\end{itemize}
\item Which value goes where
\item The FSM logic also depends on datapath output
\begin{itemize}
\item Is there enough money payed to release a can of soda?
\end{itemize}
\item FSM and datapath interact
\end{itemize}
\end{frame}


\begin{frame}[fragile]{Popcount Example}
\begin{itemize}
\item An FSMD that computes the popcount
\item Also called the Hamming weight
\item Compute the number of `1's in a word
\item Input is the data word
\item Output is the count
\item Code available at \href{https://github.com/schoeberl/chisel-book/blob/master/src/main/scala/PopCount.scala}{PopCount.scala}
\end{itemize}
\end{frame}

\begin{frame}[fragile]{Popcount Block Diagram}

\begin{figure}
  \includegraphics[scale=\scale]{../figures/popcnt-fsmd}
\end{figure}
\end{frame}


\begin{frame}[fragile]{Popcount Connection}
\begin{columns}
\column{0.6\textwidth}
\begin{itemize}
\item Input \code{din} and output \code{popCount}
\item Both connected to the datapath
\item We need some handshaking
\item For data input and for count output
\end{itemize}
\column{0.4\textwidth}
\begin{figure}
  \includegraphics[scale=0.45]{../figures/popcnt-fsmd}
\end{figure}
\end{columns}
\end{frame}

\begin{frame}[fragile]{Popcount Handshake}
\begin{columns}
\column{0.6\textwidth}
\begin{itemize}
\item We use a ready-valid handshake
\item When data is available valid is asserted
\item When the receiver can accept data ready is asserted
\item Transfer takes place when both are asserted
\end{itemize}
\column{0.4\textwidth}
\begin{figure}
  \includegraphics[scale=0.45]{../figures/popcnt-fsmd}
\end{figure}
\end{columns}
\end{frame}


\begin{frame}[fragile]{The FSM}
\begin{figure}
  \includegraphics[scale=\scale]{../figures/popcnt-states}
\end{figure}
\begin{itemize}
\item A Very Simple FSM
\item Two transitions depend on input/output handshake
\item One transition on the datapath output
\end{itemize}
\end{frame}

\begin{frame}[fragile]{xxx}
\begin{itemize}
\item yyy
\end{itemize}
\end{frame}

\begin{frame}[fragile]{xxx}
\begin{itemize}
\item yyy
\end{itemize}
\end{frame}


\begin{frame}[fragile]{xxx}
\begin{itemize}
\item yyy
\end{itemize}
\end{frame}

\begin{frame}[fragile]{Today Lab}
\begin{itemize}
\item Paper \& pencil exercises
\item Exercises on FSM
\item From the, now free, Dally book
\end{itemize}
\end{frame}

\begin{frame}[fragile]{Summary}
\begin{itemize}
\item yyy
\end{itemize}
\end{frame}


\begin{frame}[fragile]{Chisel VHDL Comparison}
\begin{columns}
\column{0.5\textwidth}
\begin{chisel}
class DecodeExecute extends Bundle {
  val rs1 = UInt(32.W)
  val rs2 = UInt(32.W)
  val immVal = UInt(32.W)
  val aluOp = new AluOp()
}
\end{chisel}
\column{0.5\textwidth}
\begin{verbatim}
VHDL code here
\end{verbatim}
\end{columns}
Also show latch and and using a button as clock
\end{frame}

\begin{frame}[fragile]{Below Stuff Moved Over}
\begin{itemize}
\item From tutorial
\item Maybe used in the future
\end{itemize}
\end{frame}



\begin{frame}[fragile]{Functional Abstraction}
\begin{chisel}
  def addSub(add: Bool, a: UInt, b: UInt) =
    Mux(add, a+b, a-b)

  val res = addSub(cond, a, b)
  
  def rising(d: Bool) = d && !RegNext(d)
\end{chisel}
\begin{itemize}
\item Functions for repeated pieces of logic
\item May contain state
\item Functions may return \emph{hardware}
\end{itemize}
\end{frame}

\begin{frame}[fragile]{Component Generation}
\begin{chisel}
val cores = new Array[Module](32)

for (j <- 0 until 32)
  cores(j) = Module(new CPU())
\end{chisel}
\begin{itemize}
\item Use Scala array to collect components
\item Generation with a Scala loop
\end{itemize}
\end{frame}

\begin{frame}[fragile]{Conditional Component Generation}
\begin{chisel}
val icache =
  if (TYPE == METHOD)
    Module(new MCache())
  else if (TYPE == LINE)
    Module(new ICache())
  else
    ChiselError.error("Unsupported Type")
\end{chisel}
\begin{itemize}
\item Use Scala if/else for conditional component types
\item Code example from Patmos
\item We parse an XML file for the configuration
\end{itemize}
\end{frame}

\begin{frame}[fragile]{Logic Generation}
\begin{itemize}
\item Read a file into a table
\begin{itemize}
\item E.g., to read in ROM content for a processor
\end{itemize}
\item Generate a truth table algorithmically
\begin{itemize}
\item E.g., generate binary to BCD translation
\end{itemize}
\item Use the full power of Scala
\end{itemize}
\begin{chisel}
val byteArray = Files.readAllBytes(Paths.get(fileName))
val arr = new Array[Bits](byteArray.length)
for (i <- 0 until byteArray.length) {
  arr(i) = Bits(byteArray(i), 8)
}
val rom = Vec[Bits](arr)
\end{chisel}
\end{frame}
%%%%%%%%%%%%%%%%%
%\begin{frame}[fragile]{zzz}
%\begin{chisel}
%code
%\end{chisel}
%\begin{itemize}
%\item xxx
%\item xxx
%\item xxx
%\end{itemize}
%\end{frame}
%
%\begin{frame}[fragile]{yyy}
%\begin{itemize}
%\item xxx
%\item xxx
%\item xxx
%\item xxx
%\item xxx
%\end{itemize}
%\end{frame}
%%%%%%%%%%%%%%%%%

\begin{frame}[fragile]{Scala Values and Variables}
\begin{chisel}
// A value is a constant
val i = 0
// No new assignment; this will not compile
i = 3

// A variable can change the value
var v = "Hello"
v = "Hello World"

// Type usually inferred, but can be declared
var s: String = "abc"
\end{chisel}
\end{frame}

\begin{frame}[fragile]{Simple Loops}
\begin{chisel}
// Loops from 0 to 9
// Automatically creates loop value i
for (i <- 0 until 10) {
  println(i)
}
\end{chisel}
\end{frame}

\begin{frame}[fragile]{Conditions}
\begin{chisel}
for (i <- 0 until 10) {
  if (i%2 == 0) {
    println(i + " is even")
  } else {
    println(i + " is odd")
  }
}
\end{chisel}
\end{frame}

\begin{frame}[fragile]{Scala Arrays and Lists}
\begin{chisel}
// An integer array with 10 elements
val numbers = new Array[Integer](10)
for (i <- 0 until numbers.length) {
  numbers(i) = i*10
}
println(numbers(9))


// List of integers
val list = List(1, 2, 3)
println(list)
// Different form of list construction
val listenum = 'a' :: 'b' :: 'c' :: Nil
println(listenum)
\end{chisel}
\end{frame}


\begin{frame}[fragile]{Scala Classes}
\begin{chisel}
// A simple class
class Example {
  // A field, initialized in the constructor
  var n = 0
  
  // A setter method
  def set(v: Integer) = {
    n = v
  }
  
  // Another method
  def print() = {
    println(n)
  }
}
\end{chisel}
\end{frame}

\begin{frame}[fragile]{Scala (Singleton) Object}
\begin{chisel}
object Example {}
\end{chisel}
\begin{itemize}
\item For \emph{static} fields and methods
\begin{itemize}
\item Scala has no static fields or methods like Java
\end{itemize}
\item Needed for \code{main}
\item Useful for helper functions
\end{itemize}
\end{frame}

\begin{frame}[fragile]{Singleton Object for the \code{main}}
\begin{chisel}
// A singleton object
object Example {
  
  // The start of a Scala program
  def main(args: Array[String]): Unit = {
    
    val e = new Example()
    e.print()
    e.set(42)
    e.print()
  }
}
\end{chisel}
\begin{itemize}
\item Compile and run it with sbt (or within Eclipse/IntelliJ):
\end{itemize}
\begin{chisel}
sbt "runMain Example"
\end{chisel}
\end{frame}

\begin{frame}[fragile]{Functions}
\begin{itemize}
\item Circuits can be encapsulated in functions
\item Each \emph{function call} generates hardware
\item Simple functions can be a single line
\end{itemize}
\begin{chisel}
  def adder(v1: UInt, v2: UInt) = v1 + v2
  
  val add1 = adder(a, b)
  val add2 = adder(c, d)
\end{chisel}
\end{frame}

\begin{frame}[fragile]{More Function Examples}
\begin{itemize}
\item Functions can also contain registers
\end{itemize}
\begin{chisel}
  def addSub(add: Bool, a: UInt, b: UInt) =
    Mux(add, a + b, a - b)

  val res = addSub(cond, a, b)

  def rising(d: Bool) = d && !RegNext(d)

  val edge = rising(cond)
\end{chisel}
\end{frame}

\begin{frame}[fragile]{The Counter as a Function}
\begin{itemize}
\item Longer functions in curly brackets
\item Last value is the return value
\end{itemize}
\begin{chisel}
def counter(n: UInt) = {
  
  val cntReg = RegInit(0.U(8.W))
  
  cntReg := cntReg + 1.U
  when(cntReg === n) {
    cntReg := 0.U
  }
  cntReg
}

val counter100 = counter(100.U)
\end{chisel}
\end{frame}


\begin{frame}[fragile]{Functions}
\begin{itemize}
\item Example from Patmos execute stage
\end{itemize}
\begin{chisel}
def alu(func: Bits, op1: UInt, op2: UInt): Bits = {
  val result = UInt(width = DATA_WIDTH)
  // some more lines...
  switch(func) {
    is(FUNC_ADD) { result := sum }
    is(FUNC_SUB) { result := op1 - op2 }
    is(FUNC_XOR) { result := (op1 ^ op2).toUInt }
    // some more lines
  }
  result
}
\end{chisel}
\end{frame}


\begin{frame}[fragile]{Parameterization}
\begin{chisel}
class ParamChannel(n: Int) extends Bundle {
  val data = Input(UInt(n.W))
  val ready = Output(Bool())
  val valid = Input(Bool())
}

val ch32 = new ParamChannel(32)
\end{chisel}
\begin{itemize}
\item Bundles and modules can be parametrized
\item Pass a parameter in the constructor
\end{itemize}

\end{frame}
\begin{frame}[fragile]{A Module with a Parameter}
\begin{chisel}
class ParamAdder(n: Int) extends Module {
  val io = IO(new Bundle {
    val a = Input(UInt(n.W))
    val b = Input(UInt(n.W))
    val result = Output(UInt(n.W))
  })

  val addVal = io.a + io.b
  io.result := addVal
}

val add8 = Module(new ParamAdder(8))
\end{chisel}
\begin{itemize}
\item Parameter can also be a Chisel type
\item Can also be a generic type:
\item \code{class Mod[T <: Bits](param: T) extends...}
\end{itemize}
\end{frame}

\begin{frame}[fragile]{Scala \code{for} Loop for Circuit Generation}
\begin{chisel}
val shiftReg = RegInit(0.U(8.W))

shiftReg(0) := inVal

for (i <- 1 until 8) {
  shiftReg(i) := shiftReg(i-1)
}
\end{chisel}
\begin{itemize}
\item \code{for} is Scala
\item This loop generates several connections
\item The connections are parallel hardware
\end{itemize}
\end{frame}

\begin{frame}[fragile]{Conditional Circuit Generation}
\begin{chisel}
class Base extends Module { val io = new Bundle() }
class VariantA extends Base { }
class VariantB extends Base { }

val m = if (useA) Module(new VariantA())
        else Module(new VariantB())
\end{chisel}
\begin{itemize}
\item \code{if} and \code{else} is Scala
\item \code{if} is an expression that returns a value
\begin{itemize}
\item Like ``\code{cond ? a : b;}'' in C and Java
\end{itemize}
\item This is not a hardware multiplexer
\item Decides which module to generate
\item Could even read an XML file for the configuration
\end{itemize}
\end{frame}

%\begin{frame}[fragile]{Chisel has a Multiplexer}
\begin{figure}
  \includegraphics[scale=\scale]{../figures/mux}
\end{figure}
\shortlist{../code/mux.txt}
\begin{itemize}
\item So what?
\item Wait... What type is \code{a} and \code{b}?
\begin{itemize}
\item Can be any Chisel type!
\end{itemize}
\end{itemize}
\end{frame}

\begin{frame}[fragile]{Chisel has a Generic Multiplexer}
\begin{figure}
  \includegraphics[scale=\scale]{../figures/mux}
\end{figure}
\shortlist{../code/mux.txt}
\begin{itemize}
\item SW people may not be impressed
\item They have generics since Java 1.5 in 2004
\begin{itemize}
\item \code{List<Flowers> != List<Cars>}
\end{itemize}
\end{itemize}
\end{frame}


\begin{frame}[fragile]{Generics in Hardware Construction}
\begin{itemize}
\item Chisel supports generic classes with type parameters
\item Write hardware generators independent of concrete type
\item This is a multiplexer \emph{generator}
\end{itemize}
\shortlist{../code/param_func.txt}
\end{frame}

\begin{frame}[fragile]{Put Generics Into Use}
\begin{itemize}
\item Let us implement a generic FIFO
\item Use the generic ready/valid interface from Chisel
\end{itemize}
\shortlist{../code/fifo_decoupled.txt}
\end{frame}

\begin{frame}[fragile]{Define the FIFO Interface}
\shortlist{../code/fifo_io.txt}
\begin{itemize}
\item We need enqueueing and dequeueing ports
\item Note the \code{Flipped}
\begin{itemize}
\item It switches the direction of ports
\item No more double definitions of an interface
\end{itemize}
\end{itemize}
\end{frame}

\begin{frame}[fragile]{But What FIFO Implementation?}
\begin{itemize}
\item Bubble FIFO (good for low data rate)
\item Double buffer FIFO (fast restart)
\item FIFO with memory and pointers (for larger buffers)
\begin{itemize}
\item Using flip-flops
\item Using on-chip memory
\end{itemize}
\item And some more...
\end{itemize}
\begin{itemize}
\item This calls for object-oriented \sout{programming} \emph{hardware construction}
\end{itemize}
\end{frame}

\begin{frame}[fragile]{Abstract Base Class and Concrete Extension}
\shortlist{../code/fifo_abstract.txt}
\begin{itemize}
\item May contain common code
\item Extend by concrete classes
\end{itemize}
\begin{chisel}
class BubbleFifo[T <: Data](gen: T, depth: Int) extends Fifo(gen: T, depth: Int) {
\end{chisel}
\end{frame}



\begin{frame}[fragile]{Select a Concrete FIFO Implementation}
\begin{itemize}
\item Decide at hardware generation
\item Can use all Scala/Java power for the decision
\begin{itemize}
\item Connect to a web service, get \sout{Google} Alphabet stock price, and decide on which to use ;-)
\item For sure a silly idea, but you see what is possible...
\item Developers may find clever use of the Scala/Java power
\item We could present a GUI to the user to select from
\end{itemize}
\item We use XML files parsed at hardware generation time
\item End of TCL, Python,... generated hardware
\end{itemize}
\end{frame}

\begin{frame}[fragile]{Binary to BCD Conversion for VHDL}
\begin{figure}
    \centering
    \includegraphics[scale=0.6]{JavaBCD}
\end{figure}
\end{frame}

\begin{frame}[fragile]{Java Program}
\begin{itemize}
\item Generates a VHDL table
\item The core code is:
\end{itemize}
\begin{chisel}
for (int i = 0; i < Math.pow(2, ADDRBITS); ++i) {
    int val = ((i/10)<<4) + i%10;
    // write out VHDL code for each line
\end{chisel}
\begin{itemize}
\item With all boilerplate 118 LoC
\end{itemize}
\end{frame}

\begin{frame}[fragile]{Chisel Version of Binary to BCD Conversion}
%\begin{chisel}
%  val array = new Array[Int](256)
%  for (i <- 0 to 99) {
%    array(i) = ((i/10)<<4) + i%10
%  }
%  val table = VecInit(array.map(_.U(8.W)))
%\end{chisel}
\begin{chisel}
  val table = Wire(Vec(100, UInt(8.W)))
  for (i <- 0 until 100) {
    table(i) := (((i/10)<<4) + i%10).U
  }
  val bcd = table(bin)
\end{chisel}
\begin{itemize}
\item Directly generates the hardware table as a \code{Vec}
\item At hardware construction time
\item In the same language
\end{itemize}
\end{frame}

\begin{frame}[fragile]{Use Functional Programming for Generators}
\shortlist{../code/fun_first.txt}
\shortlist{../code/fun_func_lit.txt}
\shortlist{../code/fun_reduce_tree.txt}
\begin{itemize}
\item This is a simple example
\item What about an arbiter tree with fair arbitration?
\end{itemize}
\end{frame}

%\begin{frame}[fragile]{XXX}
%\begin{itemize}
%\item TODO: s4noc connection is part of the generator story
%\item
%\item
%\end{itemize}
%\end{frame}
%
%\begin{frame}[fragile]{XXX}
%\begin{itemize}
%\item
%\item
%\item
%\end{itemize}
%\end{frame}



\begin{frame}[fragile]{Generation Slides (include) are missing here}
\begin{itemize}
\item 
\end{itemize}
\end{frame}

\begin{frame}[fragile]{Combinational (Truth) Table Generation}
\begin{chisel}
val arr = new Array[Bits](length)
for (i <- 0 until length) {
  arr(i) = ...
}
val rom = Vec[Bits](arr)
\end{chisel}
\begin{itemize}
\item Generate a table in a Scala array
\item Use that array as input for a Chisel \code{Vec}
\item Generates a logic table at hardware construction time
\end{itemize}
\end{frame}

\begin{frame}[fragile]{BCD Generation Example}
\begin{itemize}
\item Explain BCD with examples
\item Show code
\end{itemize}
\end{frame}

\begin{frame}[fragile]{Ideas for Runtime Table Generation}
\begin{itemize}
\item Assembler in Scala/Java generates the boot ROM
\item Table with a \code{sin} function
\item Binary to BCD conversion
\item Schedule table for a TDM based network-on-chip
\item 
\item More ideas?
\end{itemize}
\end{frame}

\begin{frame}[fragile]{Memory}
\begin{chisel}
val mem = Mem(Bits(width = 8), size)

// write
when(wrEna) {
  mem(wrAddr) := wrData
}

// read
val rdAddrReg = Reg(next = rdAddr)
rdData := mem(rdAddrReg)
\end{chisel}
\begin{itemize}
\item Write is synchronous
\item Read can be asynchronous or synchronous
\item But there are no asynchronous memories in an FPGA
\end{itemize}
\end{frame}

\begin{frame}[fragile]{Factory Methods}
\begin{itemize}
\item Simpler component creation and use
\item Usage similar to built in components, such as \code{Mux}
\end{itemize}
\begin{chisel}
val myAdder = Adder(x, y)
\end{chisel}
\begin{itemize}
\item A little bit more work on component side
\item Define an \code{apply} method on the companion object that returns the component
\end{itemize}
\begin{chisel}
object Adder {
  def apply(a: UInt, b: UInt) = {
    val adder = Module(new Adder)
    adder.io.a := a
    adder.io.b := b
    adder.io.result
  }
}
\end{chisel}
\end{frame}

\begin{frame}[fragile]{Chisel and Scala}
\begin{itemize}
\item Chisel is a library written in Scala
\begin{itemize}
\item Import the library with \code{import chisel3.\_}
\end{itemize}
\item Chisel code is Scala code
\item When it is run is \emph{generates} hardware
\begin{itemize}
\item Verilog for synthesize
\item Scala netlist for simulation (testing)
\end{itemize}
\item Chisel is an embedded domain specific language
\item Two languages in one can be a little bit confusing
\end{itemize}
\end{frame}

\begin{frame}[fragile]{Scala}
\begin{itemize}
\item Is object oriented
\item Is functional
\item Strongly typed with very good type inference
\item Runs on the Java virtual machine
\item Can call Java libraries
\item Consider it as Java++
\begin{itemize}
\item Can almost be written like Java
\item With a more lightweight syntax
% \item Scala for Java Refugees is a nice tutorial (but link is dead)
% \item \url{http://www.codecommit.com/blog/scala/roundup-scala-for-java-refugees}
\end{itemize}
\end{itemize}
\end{frame}

\begin{frame}[fragile]{Scala Hello World}
\begin{chisel}
object HelloWorld extends App {
  println("Hello, World!")
}
\end{chisel}
\begin{itemize}
\item Compile with \code{scalac} and run with \code{scala}
\item You can even use Scala as scripting language
\item Show both
\end{itemize}
\end{frame}



\begin{frame}[fragile]{Generic Components}
\begin{chisel}
val c = Mux(cond, a, b)
\end{chisel}
\begin{itemize}
\item This is a multiplexer
\item Input can be any type
\end{itemize}
\end{frame}




\begin{frame}[fragile]{A Tiny ALU: IO Connection}
\begin{chisel}
class Alu extends Module {
  val io = IO(new Bundle {
    val fn = Input(UInt(2.W))
    val a = Input(UInt(4.W))
    val b = Input(UInt(4.W))
    val result = Output(UInt(4.W))
  })

  // Use shorter variable names
  val fn = io.fn
  val a = io.a
  val b = io.b
\end{chisel}
\end{frame}

\begin{frame}[fragile]{A Tiny ALU: The Function}
\begin{chisel}
  val result = Wire(UInt(4.W))
  // some default value is needed
  result := 0.U

  // The ALU selection
  switch(fn) {
    is(0.U) { result := a + b }
    is(1.U) { result := a - b }
    is(2.U) { result := a | b }
    is(3.U) { result := a & b }
  }

  // Output on the LEDs
  io.result := result
}
\end{chisel}
\end{frame}

\begin{frame}[fragile]{Testing the ALU}
\begin{itemize}
\item Compute the expected result in Scala
\end{itemize}
\begin{chisel}
  // This is exhaustive testing,
  // which usually is impossible
  for (a <- 0 to 15) {
    for (b <- 0 to 15) {
      for (op <- 0 to 3) {
        val result =
          op match {
            case 0 => a + b
            case 1 => a - b
            case 2 => a | b
            case 3 => a & b
          }
        val resMask = result & 0x0f
\end{chisel}

\end{frame}

\begin{frame}[fragile]{Testing the ALU}
\begin{itemize}
\item Compare the Scala computed result with the hardware result
\end{itemize}
\begin{chisel}
        poke(dut.io.fn, op)
        poke(dut.io.a, a)
        poke(dut.io.b, b)
        step(1)
        expect(dut.io.result, resMask)
      }
    }
  }
\end{chisel}
\end{frame}


\begin{frame}[fragile]{Summary}
\begin{itemize}
\item Chisel is a small language
\item Embedding it in Scala gives the power
\item We can write circuit generators
\item We can to co-simulation
\item We just scratched the surface
\end{itemize}
\end{frame}


\end{document}

%\begin{frame}[fragile]{xxx}
%\begin{itemize}
%\item yyy
%\end{itemize}
%\end{frame}
